% Options for packages loaded elsewhere
\PassOptionsToPackage{unicode}{hyperref}
\PassOptionsToPackage{hyphens}{url}
%
\documentclass[
]{article}
\usepackage{lmodern}
\usepackage{amssymb,amsmath}
\usepackage{ifxetex,ifluatex}
\ifnum 0\ifxetex 1\fi\ifluatex 1\fi=0 % if pdftex
  \usepackage[T1]{fontenc}
  \usepackage[utf8]{inputenc}
  \usepackage{textcomp} % provide euro and other symbols
\else % if luatex or xetex
  \usepackage{unicode-math}
  \defaultfontfeatures{Scale=MatchLowercase}
  \defaultfontfeatures[\rmfamily]{Ligatures=TeX,Scale=1}
\fi
% Use upquote if available, for straight quotes in verbatim environments
\IfFileExists{upquote.sty}{\usepackage{upquote}}{}
\IfFileExists{microtype.sty}{% use microtype if available
  \usepackage[]{microtype}
  \UseMicrotypeSet[protrusion]{basicmath} % disable protrusion for tt fonts
}{}
\makeatletter
\@ifundefined{KOMAClassName}{% if non-KOMA class
  \IfFileExists{parskip.sty}{%
    \usepackage{parskip}
  }{% else
    \setlength{\parindent}{0pt}
    \setlength{\parskip}{6pt plus 2pt minus 1pt}}
}{% if KOMA class
  \KOMAoptions{parskip=half}}
\makeatother
\usepackage{xcolor}
\IfFileExists{xurl.sty}{\usepackage{xurl}}{} % add URL line breaks if available
\IfFileExists{bookmark.sty}{\usepackage{bookmark}}{\usepackage{hyperref}}
\hypersetup{
  pdftitle={Segundo Dia - Importando e Simulando Dados},
  pdfauthor={Pedro Cavalcante},
  hidelinks,
  pdfcreator={LaTeX via pandoc}}
\urlstyle{same} % disable monospaced font for URLs
\usepackage[margin=1in]{geometry}
\usepackage{color}
\usepackage{fancyvrb}
\newcommand{\VerbBar}{|}
\newcommand{\VERB}{\Verb[commandchars=\\\{\}]}
\DefineVerbatimEnvironment{Highlighting}{Verbatim}{commandchars=\\\{\}}
% Add ',fontsize=\small' for more characters per line
\usepackage{framed}
\definecolor{shadecolor}{RGB}{248,248,248}
\newenvironment{Shaded}{\begin{snugshade}}{\end{snugshade}}
\newcommand{\AlertTok}[1]{\textcolor[rgb]{0.94,0.16,0.16}{#1}}
\newcommand{\AnnotationTok}[1]{\textcolor[rgb]{0.56,0.35,0.01}{\textbf{\textit{#1}}}}
\newcommand{\AttributeTok}[1]{\textcolor[rgb]{0.77,0.63,0.00}{#1}}
\newcommand{\BaseNTok}[1]{\textcolor[rgb]{0.00,0.00,0.81}{#1}}
\newcommand{\BuiltInTok}[1]{#1}
\newcommand{\CharTok}[1]{\textcolor[rgb]{0.31,0.60,0.02}{#1}}
\newcommand{\CommentTok}[1]{\textcolor[rgb]{0.56,0.35,0.01}{\textit{#1}}}
\newcommand{\CommentVarTok}[1]{\textcolor[rgb]{0.56,0.35,0.01}{\textbf{\textit{#1}}}}
\newcommand{\ConstantTok}[1]{\textcolor[rgb]{0.00,0.00,0.00}{#1}}
\newcommand{\ControlFlowTok}[1]{\textcolor[rgb]{0.13,0.29,0.53}{\textbf{#1}}}
\newcommand{\DataTypeTok}[1]{\textcolor[rgb]{0.13,0.29,0.53}{#1}}
\newcommand{\DecValTok}[1]{\textcolor[rgb]{0.00,0.00,0.81}{#1}}
\newcommand{\DocumentationTok}[1]{\textcolor[rgb]{0.56,0.35,0.01}{\textbf{\textit{#1}}}}
\newcommand{\ErrorTok}[1]{\textcolor[rgb]{0.64,0.00,0.00}{\textbf{#1}}}
\newcommand{\ExtensionTok}[1]{#1}
\newcommand{\FloatTok}[1]{\textcolor[rgb]{0.00,0.00,0.81}{#1}}
\newcommand{\FunctionTok}[1]{\textcolor[rgb]{0.00,0.00,0.00}{#1}}
\newcommand{\ImportTok}[1]{#1}
\newcommand{\InformationTok}[1]{\textcolor[rgb]{0.56,0.35,0.01}{\textbf{\textit{#1}}}}
\newcommand{\KeywordTok}[1]{\textcolor[rgb]{0.13,0.29,0.53}{\textbf{#1}}}
\newcommand{\NormalTok}[1]{#1}
\newcommand{\OperatorTok}[1]{\textcolor[rgb]{0.81,0.36,0.00}{\textbf{#1}}}
\newcommand{\OtherTok}[1]{\textcolor[rgb]{0.56,0.35,0.01}{#1}}
\newcommand{\PreprocessorTok}[1]{\textcolor[rgb]{0.56,0.35,0.01}{\textit{#1}}}
\newcommand{\RegionMarkerTok}[1]{#1}
\newcommand{\SpecialCharTok}[1]{\textcolor[rgb]{0.00,0.00,0.00}{#1}}
\newcommand{\SpecialStringTok}[1]{\textcolor[rgb]{0.31,0.60,0.02}{#1}}
\newcommand{\StringTok}[1]{\textcolor[rgb]{0.31,0.60,0.02}{#1}}
\newcommand{\VariableTok}[1]{\textcolor[rgb]{0.00,0.00,0.00}{#1}}
\newcommand{\VerbatimStringTok}[1]{\textcolor[rgb]{0.31,0.60,0.02}{#1}}
\newcommand{\WarningTok}[1]{\textcolor[rgb]{0.56,0.35,0.01}{\textbf{\textit{#1}}}}
\usepackage{graphicx,grffile}
\makeatletter
\def\maxwidth{\ifdim\Gin@nat@width>\linewidth\linewidth\else\Gin@nat@width\fi}
\def\maxheight{\ifdim\Gin@nat@height>\textheight\textheight\else\Gin@nat@height\fi}
\makeatother
% Scale images if necessary, so that they will not overflow the page
% margins by default, and it is still possible to overwrite the defaults
% using explicit options in \includegraphics[width, height, ...]{}
\setkeys{Gin}{width=\maxwidth,height=\maxheight,keepaspectratio}
% Set default figure placement to htbp
\makeatletter
\def\fps@figure{htbp}
\makeatother
\setlength{\emergencystretch}{3em} % prevent overfull lines
\providecommand{\tightlist}{%
  \setlength{\itemsep}{0pt}\setlength{\parskip}{0pt}}
\setcounter{secnumdepth}{-\maxdimen} % remove section numbering

\title{Segundo Dia - Importando e Simulando Dados}
\author{Pedro Cavalcante}
\date{20 de fevereiro de 2019}

\begin{document}
\maketitle

Agora que cobrimos a sintaxe básica e sabemos conversar com o R, podemos
realmente \emph{fazer} coisas. Hoje o foco será em obter dados. Vamos
aprender a importar dados de fontes variadas, baixar da internet e
cobrir alguns pacotes que baixam dados estruturados de fontes como o
Sistema Gerador de Séries do Banco Central do Brasil, Datasus e o Sidra
do IBGE. Depois vamos aprender a gerar dados que seguem distribuições
probabilísticas e estimar distribuições com dados reais.

\hypertarget{importando-dados}{%
\section{Importando dados}\label{importando-dados}}

\hypertarget{uma-nota-cautelar}{%
\paragraph{Uma nota cautelar}\label{uma-nota-cautelar}}

Quando este material foi planejado originalmente, ainda não existia o
pacote \texttt{rio}. A seguir se segue uma explicação razoavelmente
detalhada de como usar os pacotes \texttt{readr}, \texttt{haven} e
\texttt{foreign} para ler dados das mais diversas fontes. Isso tudo pode
ser facilmente substituído pela função \texttt{import()} do pacote
\texttt{rio}. Basta por entre aspas o endereço do arquivo como argumento
e pronto.

\begin{Shaded}
\begin{Highlighting}[]
\KeywordTok{library}\NormalTok{(rio)}
\NormalTok{baseteste <-}\StringTok{ }\KeywordTok{import}\NormalTok{(}\StringTok{"exercicio3.dta"}\NormalTok{)}
\end{Highlighting}
\end{Shaded}

Se você quer ter contato com pacotes que te permitem entender melhor o
processo de importação de dados, é uma boa ideia ler a seção seguinte.
Se não, recomendo pular diretamente para a seção de simulação de dados
ou para a seção de importação de dados da internet.

Antes de prosseguirmos é importante garantir que todas as bases de
exemplos estejam baixadas e juntas numa pasta só para evitar confusões e
facilitar nosso trabalho. Mais ainda, é bom garantir que essa pasta seja
a pasta de trabalho com a função \texttt{setwd()}. Para descobrir a
pasta de trabalho atual e/ou verificar se está tudo certo, basta rodar
\texttt{getwd()}.

A maior parte dos formatos de dados é coberto pelo pacote
\texttt{readr}, então tenha-o instalado em sua máquina e carregado antes
de prosseguir.

\hypertarget{dados-delimitados-e-nas}{%
\subsection{Dados Delimitados e NAs}\label{dados-delimitados-e-nas}}

Dados delimitados são relativamente comuns. Arquivos assim contém
entradas separadas por caracteres específicos. É comum que esses
caracteres sejam um espaço, uma vírgula ou um ponto-e-vírgula.

É também comum que algumas observações não estejam completas, esse tipo
de entrada vazia é chamado de NA pelo R. Como muitas organizações usam
códigos específicos para entradas vazias - como por exemplo 999999999999
pelo IBGE na Pesquisa Nacional de Amostra Domiciliar - vamos também
aprender como fazer o R ler essas entradas como faltantes.

A função \texttt{readr::read\_delim} lê esse tipo de dado, basta
dizermos no argumento \texttt{delim\ =} qual caractere é o delimitador.
Lembre-se sempre de explorar o arquivo após a leitura para verificar se
está tudo em ordem. Para isso é comum usar a função \texttt{glimpse()}.

\begin{Shaded}
\begin{Highlighting}[]
\KeywordTok{library}\NormalTok{(readr)}
\KeywordTok{library}\NormalTok{(dplyr)}
\end{Highlighting}
\end{Shaded}

\begin{verbatim}
## 
## Attaching package: 'dplyr'
\end{verbatim}

\begin{verbatim}
## The following objects are masked from 'package:stats':
## 
##     filter, lag
\end{verbatim}

\begin{verbatim}
## The following objects are masked from 'package:base':
## 
##     intersect, setdiff, setequal, union
\end{verbatim}

\begin{Shaded}
\begin{Highlighting}[]
\NormalTok{(alturas <-}\StringTok{ }\KeywordTok{read_delim}\NormalTok{(}\StringTok{"alturas.txt"}\NormalTok{, }
                     \DataTypeTok{delim =} \StringTok{" "}\NormalTok{))}
\end{Highlighting}
\end{Shaded}

\begin{verbatim}
## Parsed with column specification:
## cols(
##   sexo = col_double(),
##   idade = col_double(),
##   altura = col_double()
## )
\end{verbatim}

\begin{verbatim}
## # A tibble: 22 x 3
##     sexo idade altura
##    <dbl> <dbl>  <dbl>
##  1     1    19    999
##  2     0    20    172
##  3     1    21    182
##  4     1    22    180
##  5     1    22    181
##  6     1   999    175
##  7     1    25    172
##  8     1    21    190
##  9     0    23    177
## 10     0    19    162
## # ... with 12 more rows
\end{verbatim}

\begin{Shaded}
\begin{Highlighting}[]
\KeywordTok{glimpse}\NormalTok{(alturas)}
\end{Highlighting}
\end{Shaded}

\begin{verbatim}
## Observations: 22
## Variables: 3
## $ sexo   <dbl> 1, 0, 1, 1, 1, 1, 1, 1, 0, 0, 1, 0, 1, 0, 1, 1, 1, 1, 0, 0, ...
## $ idade  <dbl> 19, 20, 21, 22, 22, 999, 25, 21, 23, 19, 20, 24, 22, 22, 20,...
## $ altura <dbl> 999, 172, 182, 180, 181, 175, 172, 190, 177, 162, 183, 169, ...
\end{verbatim}

Observe que o R leu duas pessoas com altura 999cm e idade 999 anos. Isso
certamente é um código para dado faltante, então precisamos reler o
arquivo, agora levando isso em conta. Esse tipo de parâmetro é
especialmente útil para lidar com microdados públicos, que quase sempre
usarão algum código para indicar a ausência do dado.

\begin{Shaded}
\begin{Highlighting}[]
\KeywordTok{library}\NormalTok{(readr)}
\NormalTok{alturas <-}\StringTok{ }\KeywordTok{read_delim}\NormalTok{(}\StringTok{"alturas.txt"}\NormalTok{, }
                     \DataTypeTok{delim =} \StringTok{" "}\NormalTok{,}
                     \DataTypeTok{na =} \StringTok{"999"}\NormalTok{)}
\end{Highlighting}
\end{Shaded}

\begin{verbatim}
## Parsed with column specification:
## cols(
##   sexo = col_double(),
##   idade = col_double(),
##   altura = col_double()
## )
\end{verbatim}

\begin{Shaded}
\begin{Highlighting}[]
\KeywordTok{glimpse}\NormalTok{(alturas)}
\end{Highlighting}
\end{Shaded}

\begin{verbatim}
## Observations: 22
## Variables: 3
## $ sexo   <dbl> 1, 0, 1, 1, 1, 1, 1, 1, 0, 0, 1, 0, 1, 0, 1, 1, 1, 1, 0, 0, ...
## $ idade  <dbl> 19, 20, 21, 22, 22, NA, 25, 21, 23, 19, 20, 24, 22, 22, 20, ...
## $ altura <dbl> NA, 172, 182, 180, 181, 175, 172, 190, 177, 162, 183, 169, 1...
\end{verbatim}

Agora sim temos uma leitura limpa dos dados.

\hypertarget{dados-de-stata-.dta-separados-por-vuxedrgula-.csv-e-de-excel-.xlsx}{%
\subsection{Dados de Stata (.dta), Separados por Vírgula (.csv) e de
Excel
(.xlsx)}\label{dados-de-stata-.dta-separados-por-vuxedrgula-.csv-e-de-excel-.xlsx}}

O processo é muito similar ao de dados delimitados. O arquivo
\texttt{ministerios.xlsx} contém dados de gastos ministeriais
brasileiros e para lê-lo vamos usar o pacote \texttt{readxl}. Os
arquivos da forma \texttt{prouni.xxx} contém dados a nível de curso do
ProUni de 2017.

Vale lembrar que cada versão do Stata tem um padrão diferente para
arquivos de dados apesar de todos serem da mesma extensão \texttt{.dta}.
Para especificar a versão, basta usar o argumento \texttt{version\ =} da
função \texttt{haven::read\_dta}. Por padrão, a versão a ser lida é a
14, mas atualmente todas entre a 8 e a 15 são permitidas.

\begin{Shaded}
\begin{Highlighting}[]
\KeywordTok{library}\NormalTok{(readxl) }\CommentTok{# para dados de Excel}
\KeywordTok{library}\NormalTok{(haven) }\CommentTok{# para dados de Stata}

\NormalTok{ministerios <-}\StringTok{ }\KeywordTok{read_xlsx}\NormalTok{(}\StringTok{"ministerios.xlsx"}\NormalTok{)}
\KeywordTok{glimpse}\NormalTok{(ministerios)}
\end{Highlighting}
\end{Shaded}

\begin{verbatim}
## Observations: 28,598
## Variables: 6
## $ `Ano/Mês` <chr> "2001/01", "2001/01", "2001/01", "2001/01", "2001/01", "2...
## $ Órgão     <chr> "CÂMARA DOS DEPUTADOS", "CÂMARA DOS DEPUTADOS", "CÂMARA D...
## $ GND       <chr> "INVESTIMENTOS", "OUTRAS DESPESAS CORRENTES", "PESSOAL E ...
## $ Pago      <dbl> 2.024110e+03, 5.596846e+06, 7.265991e+07, 1.973562e+10, 0...
## $ `RP Pago` <dbl> 919592.50, 9518239.28, 3819686.93, 72772660.10, 1870.30, ...
## $ Soma      <dbl> 9.216166e+05, 1.511509e+07, 7.647960e+07, 1.980839e+10, 1...
\end{verbatim}

\begin{Shaded}
\begin{Highlighting}[]
\NormalTok{prouni <-}\StringTok{ }\KeywordTok{read_dta}\NormalTok{(}\StringTok{"prouni.dta"}\NormalTok{)}
\KeywordTok{glimpse}\NormalTok{(prouni)}
\end{Highlighting}
\end{Shaded}

\begin{verbatim}
## Observations: 41,447
## Variables: 27
## $ Bacharelado          <dbl> 1, 1, 1, 1, 1, 1, 1, 1, 1, 1, 1, 1, 1, 1, 0, 1...
## $ Licensiatura         <dbl> 0, 0, 0, 0, 0, 0, 0, 0, 0, 0, 0, 0, 0, 0, 0, 0...
## $ Tecnologico          <dbl> 0, 0, 0, 0, 0, 0, 0, 0, 0, 0, 0, 0, 0, 0, 0, 0...
## $ Integral             <dbl> 1, 0, 1, 0, 1, 1, 1, 1, 1, 1, 1, 0, 1, 1, 0, 0...
## $ Noturno              <dbl> 0, 1, 0, 1, 0, 0, 0, 0, 0, 0, 0, 1, 0, 0, 0, 0...
## $ Matutino             <dbl> 0, 0, 0, 0, 0, 0, 0, 0, 0, 0, 0, 0, 0, 0, 1, 1...
## $ Distancia            <dbl> 0, 0, 0, 0, 0, 0, 0, 0, 0, 0, 0, 0, 0, 0, 0, 0...
## $ Vespertino           <dbl> 0, 0, 0, 0, 0, 0, 0, 0, 0, 0, 0, 0, 0, 0, 0, 0...
## $ mensalidade          <dbl> 9999.99, 9836.40, 9715.61, 9689.34, 9674.34, 9...
## $ bolsa_integral_cotas <dbl> 15, 1, 2, 3, 4, 8, 3, 4, 7, 13, 4, 5, 2, 2, 1,...
## $ bolsa_integral_ampla <dbl> 14, NA, 5, 2, 1, 15, 5, 3, 5, 4, 4, 1, 9, 1, 2...
## $ bolsa_parcial_cotas  <dbl> NA, NA, 6, NA, 5, NA, NA, NA, NA, NA, 5, NA, N...
## $ bolsa_parcial_ampla  <dbl> NA, NA, 10, NA, 2, NA, NA, NA, NA, NA, 5, NA, ...
## $ total_bolsas         <dbl> 29, 1, 23, 5, 12, 23, 8, 7, 12, 17, 18, 6, 11,...
## $ curso_id             <dbl> 7.067104e+11, 1.041912e+14, 1.002329e+12, 1.04...
## $ Medicina             <dbl> 1, 0, 1, 0, 1, 1, 1, 1, 1, 1, 1, 0, 1, 1, 0, 0...
## $ cidade_busca         <chr> "Campo Grande", "Crateus", "Sao Paulo", "Crate...
## $ uf_busca             <chr> "MS", "CE", "SP", "CE", "AC", "SP", "SP", "MG"...
## $ cidade_filtro        <chr> "NTAwMjAwNDAyNzA0", "MjMwNDAxODA0MTAz", "MzUxN...
## $ universidade_nome    <chr> "Universidade Anhanguera - UNIDERP", "Faculdad...
## $ campus_nome          <chr> "CAMPO GRANDE - SEDE - Miguel Couto", "UNIDADE...
## $ campus_id            <dbl> 706710, 1041912, 1002328, 1041912, 658996, 657...
## $ nome                 <chr> "Medicina", "Enfermagem", "Medicina", "Psicolo...
## $ nota_integral_ampla  <dbl> 740.22, 663.36, 739.62, 651.00, 758.32, 738.92...
## $ nota_integral_cotas  <dbl> 726.46, NA, 738.08, 652.22, 723.94, 728.86, 73...
## $ nota_parcial_ampla   <dbl> NA, NA, 738.96, NA, 734.92, NA, NA, NA, NA, NA...
## $ nota_parcial_cotas   <dbl> NA, NA, 718.64, NA, 711.26, NA, NA, NA, NA, NA...
\end{verbatim}

\begin{Shaded}
\begin{Highlighting}[]
\NormalTok{prouni <-}\StringTok{ }\KeywordTok{read_csv2}\NormalTok{(}\StringTok{"prouni.csv"}\NormalTok{)}
\end{Highlighting}
\end{Shaded}

\begin{verbatim}
## Using ',' as decimal and '.' as grouping mark. Use read_delim() for more control.
\end{verbatim}

\begin{verbatim}
## Parsed with column specification:
## cols(
##   .default = col_double(),
##   cidade_busca = col_character(),
##   uf_busca = col_character(),
##   cidade_filtro = col_character(),
##   universidade_nome = col_character(),
##   campus_nome = col_character(),
##   nome = col_character()
## )
\end{verbatim}

\begin{verbatim}
## See spec(...) for full column specifications.
\end{verbatim}

\begin{Shaded}
\begin{Highlighting}[]
\KeywordTok{glimpse}\NormalTok{(prouni)}
\end{Highlighting}
\end{Shaded}

\begin{verbatim}
## Observations: 41,447
## Variables: 27
## $ Bacharelado          <dbl> 1, 1, 1, 1, 1, 1, 1, 1, 1, 1, 1, 1, 1, 1, 0, 1...
## $ Licensiatura         <dbl> 0, 0, 0, 0, 0, 0, 0, 0, 0, 0, 0, 0, 0, 0, 0, 0...
## $ Tecnologico          <dbl> 0, 0, 0, 0, 0, 0, 0, 0, 0, 0, 0, 0, 0, 0, 0, 0...
## $ Integral             <dbl> 1, 0, 1, 0, 1, 1, 1, 1, 1, 1, 1, 0, 1, 1, 0, 0...
## $ Noturno              <dbl> 0, 1, 0, 1, 0, 0, 0, 0, 0, 0, 0, 1, 0, 0, 0, 0...
## $ Matutino             <dbl> 0, 0, 0, 0, 0, 0, 0, 0, 0, 0, 0, 0, 0, 0, 1, 1...
## $ Distancia            <dbl> 0, 0, 0, 0, 0, 0, 0, 0, 0, 0, 0, 0, 0, 0, 0, 0...
## $ Vespertino           <dbl> 0, 0, 0, 0, 0, 0, 0, 0, 0, 0, 0, 0, 0, 0, 0, 0...
## $ mensalidade          <dbl> 9999.99, 9836.40, 9715.61, 9689.34, 9674.34, 9...
## $ bolsa_integral_cotas <dbl> 15, 1, 2, 3, 4, 8, 3, 4, 7, 13, 4, 5, 2, 2, 1,...
## $ bolsa_integral_ampla <dbl> 14, NA, 5, 2, 1, 15, 5, 3, 5, 4, 4, 1, 9, 1, 2...
## $ bolsa_parcial_cotas  <dbl> NA, NA, 6, NA, 5, NA, NA, NA, NA, NA, 5, NA, N...
## $ bolsa_parcial_ampla  <dbl> NA, NA, 10, NA, 2, NA, NA, NA, NA, NA, 5, NA, ...
## $ total_bolsas         <dbl> 29, 1, 23, 5, 12, 23, 8, 7, 12, 17, 18, 6, 11,...
## $ curso_id             <dbl> 7.067104e+11, 1.041912e+14, 1.002329e+12, 1.04...
## $ Medicina             <dbl> 1, 0, 1, 0, 1, 1, 1, 1, 1, 1, 1, 0, 1, 1, 0, 0...
## $ cidade_busca         <chr> "Campo Grande", "Crateus", "Sao Paulo", "Crate...
## $ uf_busca             <chr> "MS", "CE", "SP", "CE", "AC", "SP", "SP", "MG"...
## $ cidade_filtro        <chr> "NTAwMjAwNDAyNzA0", "MjMwNDAxODA0MTAz", "MzUxN...
## $ universidade_nome    <chr> "Universidade Anhanguera - UNIDERP", "Faculdad...
## $ campus_nome          <chr> "CAMPO GRANDE - SEDE - Miguel Couto", "UNIDADE...
## $ campus_id            <dbl> 706710, 1041912, 1002328, 1041912, 658996, 657...
## $ nome                 <chr> "Medicina", "Enfermagem", "Medicina", "Psicolo...
## $ nota_integral_ampla  <dbl> 740.22, 663.36, 739.62, 651.00, 758.32, 738.92...
## $ nota_integral_cotas  <dbl> 726.46, NA, 738.08, 652.22, 723.94, 728.86, 73...
## $ nota_parcial_ampla   <dbl> NA, NA, 738.96, NA, 734.92, NA, NA, NA, NA, NA...
## $ nota_parcial_cotas   <dbl> NA, NA, 718.64, NA, 711.26, NA, NA, NA, NA, NA...
\end{verbatim}

\hypertarget{formatos-especuxedficos-do-r}{%
\subsection{Formatos específicos do
R}\label{formatos-especuxedficos-do-r}}

A linguagem R tem duas extensões de dados, \texttt{.Rdata} e
\texttt{.Rds}. Ambas são muito úteis porque geram arquivos bem mais
leves para um mesmo volume de dados que vários outros formatos
populares. \texttt{.Rdata} serve para armazenar vários objetos de R -
como por exemplo todo o seu ambiente de trabalho - e \texttt{.Rds} para
um único objeto. Usamos as funções \texttt{readRDS()} para ler arquivos
dessa extensão e \texttt{writeRDS()} para salvar arquivos.

\begin{Shaded}
\begin{Highlighting}[]
\NormalTok{prouni <-}\StringTok{ }\KeywordTok{readRDS}\NormalTok{(}\StringTok{"prouni.Rds"}\NormalTok{)}
\KeywordTok{glimpse}\NormalTok{(prouni)}
\end{Highlighting}
\end{Shaded}

\begin{verbatim}
## Observations: 41,447
## Variables: 27
## $ Bacharelado          <dbl> 1, 1, 1, 1, 1, 1, 1, 1, 1, 1, 1, 1, 1, 1, 0, 1...
## $ Licensiatura         <dbl> 0, 0, 0, 0, 0, 0, 0, 0, 0, 0, 0, 0, 0, 0, 0, 0...
## $ Tecnologico          <dbl> 0, 0, 0, 0, 0, 0, 0, 0, 0, 0, 0, 0, 0, 0, 0, 0...
## $ Integral             <dbl> 1, 0, 1, 0, 1, 1, 1, 1, 1, 1, 1, 0, 1, 1, 0, 0...
## $ Noturno              <dbl> 0, 1, 0, 1, 0, 0, 0, 0, 0, 0, 0, 1, 0, 0, 0, 0...
## $ Matutino             <dbl> 0, 0, 0, 0, 0, 0, 0, 0, 0, 0, 0, 0, 0, 0, 1, 1...
## $ Distancia            <dbl> 0, 0, 0, 0, 0, 0, 0, 0, 0, 0, 0, 0, 0, 0, 0, 0...
## $ Vespertino           <dbl> 0, 0, 0, 0, 0, 0, 0, 0, 0, 0, 0, 0, 0, 0, 0, 0...
## $ mensalidade          <dbl> 9999.99, 9836.40, 9715.61, 9689.34, 9674.34, 9...
## $ bolsa_integral_cotas <dbl> 15, 1, 2, 3, 4, 8, 3, 4, 7, 13, 4, 5, 2, 2, 1,...
## $ bolsa_integral_ampla <dbl> 14, NA, 5, 2, 1, 15, 5, 3, 5, 4, 4, 1, 9, 1, 2...
## $ bolsa_parcial_cotas  <dbl> NA, NA, 6, NA, 5, NA, NA, NA, NA, NA, 5, NA, N...
## $ bolsa_parcial_ampla  <dbl> NA, NA, 10, NA, 2, NA, NA, NA, NA, NA, 5, NA, ...
## $ total_bolsas         <dbl> 29, 1, 23, 5, 12, 23, 8, 7, 12, 17, 18, 6, 11,...
## $ curso_id             <dbl> 7.067104e+11, 1.041912e+14, 1.002329e+12, 1.04...
## $ Medicina             <dbl> 1, 0, 1, 0, 1, 1, 1, 1, 1, 1, 1, 0, 1, 1, 0, 0...
## $ cidade_busca         <chr> "Campo Grande", "Crateus", "Sao Paulo", "Crate...
## $ uf_busca             <chr> "MS", "CE", "SP", "CE", "AC", "SP", "SP", "MG"...
## $ cidade_filtro        <chr> "NTAwMjAwNDAyNzA0", "MjMwNDAxODA0MTAz", "MzUxN...
## $ universidade_nome    <chr> "Universidade Anhanguera - UNIDERP", "Faculdad...
## $ campus_nome          <chr> "CAMPO GRANDE - SEDE - Miguel Couto", "UNIDADE...
## $ campus_id            <dbl> 706710, 1041912, 1002328, 1041912, 658996, 657...
## $ nome                 <chr> "Medicina", "Enfermagem", "Medicina", "Psicolo...
## $ nota_integral_ampla  <dbl> 740.22, 663.36, 739.62, 651.00, 758.32, 738.92...
## $ nota_integral_cotas  <dbl> 726.46, NA, 738.08, 652.22, 723.94, 728.86, 73...
## $ nota_parcial_ampla   <dbl> NA, NA, 738.96, NA, 734.92, NA, NA, NA, NA, NA...
## $ nota_parcial_cotas   <dbl> NA, NA, 718.64, NA, 711.26, NA, NA, NA, NA, NA...
\end{verbatim}

\hypertarget{dados-da-internet}{%
\subsection{Dados da Internet}\label{dados-da-internet}}

\hypertarget{de-links-diretos}{%
\subsubsection{De links Diretos}\label{de-links-diretos}}

Uma ferramenta muito boa em R são arquivos temporários. Eles podem ser
usados para diversos fins e baixar arquivos diretamente de links dados é
um dos mais interessantes. Esse método envolve certa ``malícia'' de
localizar um link correto de download. Não basta propriamento informar a
URL de uma página que contenha o link de download do arquivo, precisamos
informar o link em si.

Depois de localizado o link para o download do arquivo, o procedimento
que normalmente funciona é clicar com o botão direito e depois em
``Copiar o Endereço do Link''. Esse endereço é o que usaremos.

\begin{figure}
\centering
\includegraphics{https://i.imgur.com/w1gdU5j.png}
\caption{``View Raw'' no Github}
\end{figure}

\begin{figure}
\centering
\includegraphics{https://i.imgur.com/HfxAj3y.png}
\caption{Copiando o endereço do link de uma página da AER}
\end{figure}

O procedimento é razoavelmente simples. Primeiro definimos um arquivo
temporário com a função \texttt{tempfile()}, depois baixamos o arquivo
usando a função \texttt{download.file()} - cuidado ao alterar o
parâmetro \texttt{mode=}, ele é um tanto quanto imprevisível - e depois
lemos com a função apropriada o arquivo baixado. Como estamos baixando
um arquivo em extensão \texttt{.rds}, devemos usar \texttt{readRDS()}.

\begin{Shaded}
\begin{Highlighting}[]
\NormalTok{link <-}\StringTok{ "https://github.com/danmrc/azul/blob/master/content/post/cox_rais/acre_rais_2017.Rds?raw=true"}

\NormalTok{temporario <-}\StringTok{ }\KeywordTok{tempfile}\NormalTok{()}
\KeywordTok{download.file}\NormalTok{(link, }\DataTypeTok{destfile =}\NormalTok{ temporario, }\DataTypeTok{mode =} \StringTok{"wb"}\NormalTok{)}

\NormalTok{RAIS_acre <-}\StringTok{ }\KeywordTok{readRDS}\NormalTok{(temporario)}
\KeywordTok{glimpse}\NormalTok{(RAIS_acre)}
\end{Highlighting}
\end{Shaded}

\hypertarget{de-pacotes}{%
\subsubsection{De Pacotes}\label{de-pacotes}}

Vários pacotes de R trazem funcionalidades para importar dados de
maneira mais simples, vamos cobrir alguns aqui.

\hypertarget{brazilian-economic-time-series-bets}{%
\paragraph{Brazilian Economic Time Series
(BETS)}\label{brazilian-economic-time-series-bets}}

O BETS extrai séries diretamente do SGS/BCB, basta alimentar um código
específico da série. A função \texttt{BETSsearch()} permite fazer
buscas, mas o código também pode ser manualmente localizado no site do
SGS:
\url{https://www3.bcb.gov.br/sgspub/localizarseries/localizarSeries.do?method=prepararTelaLocalizarSeries}.

Como spoiler: o código da série de variação mensal do IPCA é 433. Vamos
importar uma série e gerar um gráfico. Não se preocupe se você ainda não
sabe fazer gráficos no R. Vamos ver isso no próximo encontro - e
aprenderemos a fazer gráficos muito mais bonitos.

\begin{Shaded}
\begin{Highlighting}[]
\KeywordTok{library}\NormalTok{(BETS)}
\KeywordTok{citation}\NormalTok{(}\StringTok{"BETS"}\NormalTok{)}

\KeywordTok{BETSsearch}\NormalTok{(}\DataTypeTok{description =} \StringTok{"ipca"}\NormalTok{)}

\NormalTok{ipca <-}\StringTok{ }\KeywordTok{BETSget}\NormalTok{(}\DecValTok{433}\NormalTok{)}
\KeywordTok{plot}\NormalTok{(ipca, }
     \DataTypeTok{main =} \StringTok{"Variação Mensal do IPCA, 1980-2019"}\NormalTok{, }
     \DataTypeTok{ylab =} \StringTok{"%"}\NormalTok{,}
     \DataTypeTok{xlab =} \StringTok{"Ano"}\NormalTok{,}
     \DataTypeTok{sub =} \StringTok{"Dados do Banco Central do Brasil"}\NormalTok{)}
\end{Highlighting}
\end{Shaded}

\hypertarget{electionsbr}{%
\paragraph{ElectionsBR}\label{electionsbr}}

O pacote \texttt{electionsBR} traz ferramentas para importar dados
eleitorais do repositório do TSE. É uma boa ideia olhar a documentação
do pacote porque cada tipo de eleição tem uma função própria. Como
exemplo vou usar \texttt{president\_mun\_vote()} para baixar dados de
eleição presidencial a nível de municípios.

\begin{Shaded}
\begin{Highlighting}[]
\KeywordTok{library}\NormalTok{(electionsBR)}
\KeywordTok{citation}\NormalTok{(}\StringTok{"electionsBR"}\NormalTok{)}

\NormalTok{presidente <-}\StringTok{ }\KeywordTok{president_mun_vote}\NormalTok{(}\DataTypeTok{year =} \DecValTok{2010}\NormalTok{)}
\KeywordTok{glimpse}\NormalTok{(presidente)}
\end{Highlighting}
\end{Shaded}

\hypertarget{outros-pacotes}{%
\paragraph{Outros pacotes}\label{outros-pacotes}}

À essa altura espero que você já tenha entendido um pouco o espírito da
coisa, então vou deixar uma lista pequena de outros pacotes úteis para
importar dados.

\begin{itemize}
\tightlist
\item
  \href{https://www.quandl.com/tools/r}{Quandl} é ótimo para importar
  dados financeiros.
\item
  \href{https://github.com/danicat/datasus}{datasus, da Daniela
  Petruzalek}. Existem vários pacotes com esse mesmo nome e propósito,
  mas o da Daniela me pareceu o melhor e mais funcional.
\item
  \href{https://github.com/lucasmation/microdadosBrasil}{microdadosBrasil}
  importa e trata varios microdados brasileiros como Censo, PNAD, RAIS e
  Censo do Ensino Superior.
\item
  \href{https://cran.r-project.org/web/packages/BatchGetSymbols/BatchGetSymbols.pdf}{BatchGetSymbols}
  ajuda muito a lidar com dados financeiros e importa-los corretamente.
\item
  \href{https://www.quantmod.com/}{quantmod} disponibiliza um ambiente
  para construção de modelos e cenários, outro muito útil para lidar com
  dados financeiros.
\end{itemize}

\hypertarget{simulando-dados}{%
\section{Simulando dados}\label{simulando-dados}}

Para simular dados no R basta especificarmos uma distribuição
probabilística. O pacote \texttt{stats}, parte da Biblioteca Padrão,
traz várias funções que cobrem todas as facetas das distribuições mais
comuns, como a Normal, Poisson, binomial e exponencial. Distribuições
mais exóticas como a Pareto II, Pareto IV, de Maddala e Lognormal
Discreta estão disponíveis em pacotes como o \texttt{VGAM} e o
\texttt{extraDistr}. Para uma lista completa dos pacotes disponíveis com
distribuições, é só acessar o CRAN TaskView de Distribuições
Probabilísticas, disponível em:
\url{https://cran.r-project.org/web/views/Distributions.html}.

Funções relacionadas à distribuições sempre têm a mesma estrutura de
nome: \texttt{xfoo}, onde x é uma letra e foo é abreviação de alguma
distribuição. As letras são, normalmente, \texttt{d}, \texttt{p},
\texttt{q} e \texttt{r}, cada uma com um propósito específico.

\begin{itemize}
\tightlist
\item
  Funções \texttt{p} devolvem a probabilidade acumulada de uma entrada.
  É a Função de Distribuição Acumulada.
\item
  Funções \texttt{q} devolvem o quantil de uma probabilidade
  especificada. É a Função Inversa de Distribuição Acumulada.
\item
  Funções \texttt{d} devolvem a densidade da distribuição, a derivada da
  Função de Distribuição Acumulada.
\item
  Funções \texttt{r} (do inglês, \emph{random}) devolvem um valor
  aleatório que segue a distribuição especificada.
\end{itemize}

Algumas funções e suas abreviações:

\begin{itemize}
\tightlist
\item
  \texttt{norm} para a Normal
\item
  \texttt{pois} para a Poisson
\item
  \texttt{t} para a T de Student
\item
  \texttt{exp} para a exponencial
\item
  \texttt{gamma} para a Gamma
\item
  \texttt{f} para a F de Fisher-Snedecor
\item
  \texttt{unif} para a Uniforme
\end{itemize}

\hypertarget{um-pequeno-aviso-sobre-gerauxe7uxe3o-de-dados-aleatuxf3rios}{%
\subsection{Um pequeno aviso sobre geração de dados
``aleatórios''}\label{um-pequeno-aviso-sobre-gerauxe7uxe3o-de-dados-aleatuxf3rios}}

É difícil fielmente gerar dados \emph{aleatórios} porque sempre
precisamos de algum processo gerador dos dados por trás. Esse assunto é
um tanto quanto complexo, mas vale introduzir uma noção aqui de Semente
Aleatória (Random Seed).

Todo gerador pseudoaleatório de números precisa de um número inicial
para começar. Chamamos esse número de Semente (Aleatória) ou de Seed. O
R tem uma semente padrão, mas para garantir que nossos resultados não
mudam entre uma sessão de trabalho e outra, é sempre de bom tom usar a
função \texttt{set.seed()} para definir uma semente antes de começar a
trabalhar e deixa-la em nossos scripts. O número em si não importa, pode
ser 1, 123, 1001 ou 6516871651567416871.

Bons geradores pseudoaleatórios de números são pouco sensíveis à escolha
da semente, só precisam de um número qualquer. É bom rodar seus modelos
com 2 ou 3 sementes diferentes para verificar que seus resultados não
estão aparecendo por conta da semente escolhida, mas isso tem como
objetivo aliviar a consciência muito mais do que encontrar erros. É
realmente difícil que uma semente específica introduza resultados muito
peculiares.

Tenha em mente, no entanto, que um mesmo processo pseudoaleatório gera
os mesmos dados se tiver a mesma semente. Em alguns momentos isso é
interessante, em outros não. Vou dar dois exemplos:

\begin{itemize}
\item
  Imagine que você está rodando alguns modelos de classificação de risco
  para uma seguradora. É importante garantir consistência sempre, então
  escolher a \emph{mesma} semente é uma boa prática, dado que você já
  testou o modelo com algumas sementes diferentes.
\item
  Imagine que você tem um bot de twitter que tweeta aleatoriamente
  pedaços de letras de um artista, escolhidas ao acaso de um banco de
  dados com letras desse artista. Para não repetir a mesma escolha
  sempre, é bom tornar o processo de escolher a semente aleatório. Um
  bom truque é usar a função \texttt{Sys.time()}, que pega data e hora
  no momento em que é rodada, combinar com a função
  \texttt{as.numeric()} para transforma-la em um número e então dar esse
  número como semente aleatória.
\end{itemize}

\begin{Shaded}
\begin{Highlighting}[]
\KeywordTok{set.seed}\NormalTok{(}\KeywordTok{as.numeric}\NormalTok{(}\KeywordTok{Sys.time}\NormalTok{()))}
\end{Highlighting}
\end{Shaded}

É também importante ter em mente quais são os argumentos padrão das
funções sendo usadas. A função \texttt{rnorm}, por exemplo, tem por
padrão média 0 e desvio-padrão unitário - porque essa é a Normal Padrão.
Lembre-se de que as funções para simular distribuições no R sempre
recebem como argumento o \emph{desvio-padrão}, no argumento \texttt{sd=}
e não a variância. Já as funções da distribuição Gamma aceitam duas
fórmulas diferentes para a densidade da distribuição, então é importante
ler a documentação para saber como conseguir a que você quer.

\begin{Shaded}
\begin{Highlighting}[]
\KeywordTok{set.seed}\NormalTok{(}\DecValTok{1234}\NormalTok{) }\CommentTok{# semente definida}

\KeywordTok{pnorm}\NormalTok{(}\DecValTok{28}\NormalTok{, }\DataTypeTok{mean =} \DecValTok{25}\NormalTok{, }\DataTypeTok{sd =} \KeywordTok{sqrt}\NormalTok{(}\DecValTok{32}\NormalTok{)) }\CommentTok{# probabilidade acumulada do valor 28 numa distribuição N(25, 32)}
\end{Highlighting}
\end{Shaded}

\begin{verbatim}
## [1] 0.7020585
\end{verbatim}

\begin{Shaded}
\begin{Highlighting}[]
\KeywordTok{rnorm}\NormalTok{(}\DataTypeTok{n =} \DecValTok{10}\NormalTok{) }\CommentTok{# 10 números aleatórios com distribuição de uma normal padrão}
\end{Highlighting}
\end{Shaded}

\begin{verbatim}
##  [1] -1.2070657  0.2774292  1.0844412 -2.3456977  0.4291247  0.5060559
##  [7] -0.5747400 -0.5466319 -0.5644520 -0.8900378
\end{verbatim}

\begin{Shaded}
\begin{Highlighting}[]
\KeywordTok{rnorm}\NormalTok{(}\DecValTok{15}\NormalTok{, }\DataTypeTok{mean =} \DecValTok{2}\NormalTok{, }\DataTypeTok{sd =} \DecValTok{2}\NormalTok{) }\CommentTok{# 15 números aleatórios com distribuição de uma normal N(2, 4)}
\end{Highlighting}
\end{Shaded}

\begin{verbatim}
##  [1] 1.04561460 0.00322711 0.44749221 2.12891763 3.91898812 1.77942901
##  [7] 0.97798099 0.17760917 0.32565664 6.83167036 2.26817644 1.01862821
## [13] 1.11890426 2.91917888 0.61255951
\end{verbatim}

\begin{Shaded}
\begin{Highlighting}[]
\KeywordTok{qnorm}\NormalTok{(}\FloatTok{0.3}\NormalTok{, }\DataTypeTok{mean =} \DecValTok{25}\NormalTok{, }\DataTypeTok{sd =} \KeywordTok{sqrt}\NormalTok{(}\DecValTok{32}\NormalTok{)) }\CommentTok{# número com probabilidade acumulada de 30% em uma normal N(25, 32)}
\end{Highlighting}
\end{Shaded}

\begin{verbatim}
## [1] 22.03354
\end{verbatim}

\begin{Shaded}
\begin{Highlighting}[]
\KeywordTok{dt}\NormalTok{(}\DecValTok{2}\NormalTok{, }\DataTypeTok{df =} \DecValTok{10}\NormalTok{) }\CommentTok{# valor da densidade de uma t de Student com 10 graus de liberdade no 2}
\end{Highlighting}
\end{Shaded}

\begin{verbatim}
## [1] 0.06114577
\end{verbatim}

\begin{Shaded}
\begin{Highlighting}[]
\KeywordTok{punif}\NormalTok{(}\DecValTok{0}\NormalTok{, }\DataTypeTok{min =} \DecValTok{-1}\NormalTok{, }\DataTypeTok{max =} \DecValTok{1}\NormalTok{) }\CommentTok{# probabilidade acumulada no 0 de uma U(-1, 1)}
\end{Highlighting}
\end{Shaded}

\begin{verbatim}
## [1] 0.5
\end{verbatim}

Podemos testar propriedades de dados para verificar se aderem à
distribuições particulares com alguns testes estatísticos. Tome
\(X \sim N(2, 8)\). Vamos gerar uma amostra com \(n = 2000\) que segue a
distribuição \(X\). Depois, vamos aplicar dois testes de hipótese muito
comuns. O primeiro é o de Shapiro-Wilk, serve para testar normalidade e
o outro é o de Kolmogorof-Smirnov, que serve para testar se dados de uma
amostra aderem a uma distribuição contínua qualquer.

\begin{Shaded}
\begin{Highlighting}[]
\NormalTok{X <-}\StringTok{ }\KeywordTok{rnorm}\NormalTok{(}\DataTypeTok{n =} \DecValTok{2000}\NormalTok{,}
          \DataTypeTok{mean =} \DecValTok{2}\NormalTok{,}
          \DataTypeTok{sd =} \KeywordTok{sqrt}\NormalTok{(}\DecValTok{8}\NormalTok{))}

\KeywordTok{sd}\NormalTok{(X) }\CommentTok{# desvio-padrão dos dados}
\end{Highlighting}
\end{Shaded}

\begin{verbatim}
## [1] 2.802836
\end{verbatim}

\begin{Shaded}
\begin{Highlighting}[]
\KeywordTok{var}\NormalTok{(X) }\CommentTok{# variância dos dados}
\end{Highlighting}
\end{Shaded}

\begin{verbatim}
## [1] 7.855889
\end{verbatim}

\begin{Shaded}
\begin{Highlighting}[]
\KeywordTok{mean}\NormalTok{(X) }\CommentTok{# média dos dados}
\end{Highlighting}
\end{Shaded}

\begin{verbatim}
## [1] 1.986049
\end{verbatim}

\begin{Shaded}
\begin{Highlighting}[]
\KeywordTok{sample}\NormalTok{(X, }\DataTypeTok{size =} \DecValTok{7}\NormalTok{) }\CommentTok{# 7 elementos aleatoriamente escolhidos de X}
\end{Highlighting}
\end{Shaded}

\begin{verbatim}
## [1]  1.4069975 -3.2296778 -0.9301134  7.0824336 -0.8883707  2.4811205  4.1747019
\end{verbatim}

\begin{Shaded}
\begin{Highlighting}[]
\KeywordTok{shapiro.test}\NormalTok{(X) }\CommentTok{# teste de Shapiro-Wilk de normalidade de dados}
\end{Highlighting}
\end{Shaded}

\begin{verbatim}
## 
##  Shapiro-Wilk normality test
## 
## data:  X
## W = 0.99915, p-value = 0.4939
\end{verbatim}

\begin{Shaded}
\begin{Highlighting}[]
\KeywordTok{ks.test}\NormalTok{(X, }
        \StringTok{"pnorm"}\NormalTok{, }
        \DataTypeTok{mean =} \DecValTok{2}\NormalTok{, }
        \DataTypeTok{sd =} \KeywordTok{sqrt}\NormalTok{(}\DecValTok{8}\NormalTok{)) }\CommentTok{# teste de Kolmogorov-Smirnov de Aderência}
\end{Highlighting}
\end{Shaded}

\begin{verbatim}
## 
##  One-sample Kolmogorov-Smirnov test
## 
## data:  X
## D = 0.013611, p-value = 0.8526
## alternative hypothesis: two-sided
\end{verbatim}

\hypertarget{exercuxedcios}{%
\section{Exercícios}\label{exercuxedcios}}

\begin{itemize}
\tightlist
\item
  Abra o arquivo \texttt{exercicio1.Rds} e teste se ele adere à uma
  distribuição t de Student com 5, 10 ou 15 graus de liberdade.
\item
  Abra o arquivo \texttt{exercicio2.Rds} e teste se os dados aderem à
  uma Normal com média 0 e variância 10.
\item
  Abra o arquivo \texttt{exercicio3.dta} e teste se os dados são
  normais.
\item
  \(P(3 < Y < 30)\) dado que \(Y \sim \text{Poisson}(8)\).
\item
  Gere uma amostra com \(n=500\) da variável \(A \sim F(2, 4)\).
\item
  Encontre o número \(c\) tal que \(P(Z>c) = 0.7\) e
  \(Z \sim Gamma(1, 5)\).
\end{itemize}

\end{document}
