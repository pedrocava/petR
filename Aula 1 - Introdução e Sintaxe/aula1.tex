% Options for packages loaded elsewhere
\PassOptionsToPackage{unicode}{hyperref}
\PassOptionsToPackage{hyphens}{url}
%
\documentclass[
]{article}
\usepackage{lmodern}
\usepackage{amssymb,amsmath}
\usepackage{ifxetex,ifluatex}
\ifnum 0\ifxetex 1\fi\ifluatex 1\fi=0 % if pdftex
  \usepackage[T1]{fontenc}
  \usepackage[utf8]{inputenc}
  \usepackage{textcomp} % provide euro and other symbols
\else % if luatex or xetex
  \usepackage{unicode-math}
  \defaultfontfeatures{Scale=MatchLowercase}
  \defaultfontfeatures[\rmfamily]{Ligatures=TeX,Scale=1}
\fi
% Use upquote if available, for straight quotes in verbatim environments
\IfFileExists{upquote.sty}{\usepackage{upquote}}{}
\IfFileExists{microtype.sty}{% use microtype if available
  \usepackage[]{microtype}
  \UseMicrotypeSet[protrusion]{basicmath} % disable protrusion for tt fonts
}{}
\makeatletter
\@ifundefined{KOMAClassName}{% if non-KOMA class
  \IfFileExists{parskip.sty}{%
    \usepackage{parskip}
  }{% else
    \setlength{\parindent}{0pt}
    \setlength{\parskip}{6pt plus 2pt minus 1pt}}
}{% if KOMA class
  \KOMAoptions{parskip=half}}
\makeatother
\usepackage{xcolor}
\IfFileExists{xurl.sty}{\usepackage{xurl}}{} % add URL line breaks if available
\IfFileExists{bookmark.sty}{\usepackage{bookmark}}{\usepackage{hyperref}}
\hypersetup{
  pdftitle={Primeiro Dia - Introdução e Sintaxe},
  pdfauthor={Pedro Cavalcante},
  hidelinks,
  pdfcreator={LaTeX via pandoc}}
\urlstyle{same} % disable monospaced font for URLs
\usepackage[margin=1in]{geometry}
\usepackage{color}
\usepackage{fancyvrb}
\newcommand{\VerbBar}{|}
\newcommand{\VERB}{\Verb[commandchars=\\\{\}]}
\DefineVerbatimEnvironment{Highlighting}{Verbatim}{commandchars=\\\{\}}
% Add ',fontsize=\small' for more characters per line
\usepackage{framed}
\definecolor{shadecolor}{RGB}{248,248,248}
\newenvironment{Shaded}{\begin{snugshade}}{\end{snugshade}}
\newcommand{\AlertTok}[1]{\textcolor[rgb]{0.94,0.16,0.16}{#1}}
\newcommand{\AnnotationTok}[1]{\textcolor[rgb]{0.56,0.35,0.01}{\textbf{\textit{#1}}}}
\newcommand{\AttributeTok}[1]{\textcolor[rgb]{0.77,0.63,0.00}{#1}}
\newcommand{\BaseNTok}[1]{\textcolor[rgb]{0.00,0.00,0.81}{#1}}
\newcommand{\BuiltInTok}[1]{#1}
\newcommand{\CharTok}[1]{\textcolor[rgb]{0.31,0.60,0.02}{#1}}
\newcommand{\CommentTok}[1]{\textcolor[rgb]{0.56,0.35,0.01}{\textit{#1}}}
\newcommand{\CommentVarTok}[1]{\textcolor[rgb]{0.56,0.35,0.01}{\textbf{\textit{#1}}}}
\newcommand{\ConstantTok}[1]{\textcolor[rgb]{0.00,0.00,0.00}{#1}}
\newcommand{\ControlFlowTok}[1]{\textcolor[rgb]{0.13,0.29,0.53}{\textbf{#1}}}
\newcommand{\DataTypeTok}[1]{\textcolor[rgb]{0.13,0.29,0.53}{#1}}
\newcommand{\DecValTok}[1]{\textcolor[rgb]{0.00,0.00,0.81}{#1}}
\newcommand{\DocumentationTok}[1]{\textcolor[rgb]{0.56,0.35,0.01}{\textbf{\textit{#1}}}}
\newcommand{\ErrorTok}[1]{\textcolor[rgb]{0.64,0.00,0.00}{\textbf{#1}}}
\newcommand{\ExtensionTok}[1]{#1}
\newcommand{\FloatTok}[1]{\textcolor[rgb]{0.00,0.00,0.81}{#1}}
\newcommand{\FunctionTok}[1]{\textcolor[rgb]{0.00,0.00,0.00}{#1}}
\newcommand{\ImportTok}[1]{#1}
\newcommand{\InformationTok}[1]{\textcolor[rgb]{0.56,0.35,0.01}{\textbf{\textit{#1}}}}
\newcommand{\KeywordTok}[1]{\textcolor[rgb]{0.13,0.29,0.53}{\textbf{#1}}}
\newcommand{\NormalTok}[1]{#1}
\newcommand{\OperatorTok}[1]{\textcolor[rgb]{0.81,0.36,0.00}{\textbf{#1}}}
\newcommand{\OtherTok}[1]{\textcolor[rgb]{0.56,0.35,0.01}{#1}}
\newcommand{\PreprocessorTok}[1]{\textcolor[rgb]{0.56,0.35,0.01}{\textit{#1}}}
\newcommand{\RegionMarkerTok}[1]{#1}
\newcommand{\SpecialCharTok}[1]{\textcolor[rgb]{0.00,0.00,0.00}{#1}}
\newcommand{\SpecialStringTok}[1]{\textcolor[rgb]{0.31,0.60,0.02}{#1}}
\newcommand{\StringTok}[1]{\textcolor[rgb]{0.31,0.60,0.02}{#1}}
\newcommand{\VariableTok}[1]{\textcolor[rgb]{0.00,0.00,0.00}{#1}}
\newcommand{\VerbatimStringTok}[1]{\textcolor[rgb]{0.31,0.60,0.02}{#1}}
\newcommand{\WarningTok}[1]{\textcolor[rgb]{0.56,0.35,0.01}{\textbf{\textit{#1}}}}
\usepackage{graphicx,grffile}
\makeatletter
\def\maxwidth{\ifdim\Gin@nat@width>\linewidth\linewidth\else\Gin@nat@width\fi}
\def\maxheight{\ifdim\Gin@nat@height>\textheight\textheight\else\Gin@nat@height\fi}
\makeatother
% Scale images if necessary, so that they will not overflow the page
% margins by default, and it is still possible to overwrite the defaults
% using explicit options in \includegraphics[width, height, ...]{}
\setkeys{Gin}{width=\maxwidth,height=\maxheight,keepaspectratio}
% Set default figure placement to htbp
\makeatletter
\def\fps@figure{htbp}
\makeatother
\setlength{\emergencystretch}{3em} % prevent overfull lines
\providecommand{\tightlist}{%
  \setlength{\itemsep}{0pt}\setlength{\parskip}{0pt}}
\setcounter{secnumdepth}{-\maxdimen} % remove section numbering

\title{Primeiro Dia - Introdução e Sintaxe}
\author{Pedro Cavalcante}
\date{20 de fevereiro de 2019}

\begin{document}
\maketitle

\hypertarget{o-que-uxe9-r}{%
\section{O que é R}\label{o-que-uxe9-r}}

R é uma linguagem de programação voltada para estatística e análise de
dados \emph{open-source}, gratuita e de natureza colaborativa. Como
vamos ver, o ambiente voltado para análise de dados facilita enormemente
as tarefas do ciclo.

\begin{figure}
\includegraphics[width=1\linewidth]{data-science} \caption{A caption}\label{fig:unnamed-chunk-1}
\end{figure}

Instalar o R é muito simples, ele está disponível em
\url{http://cran.r-project.org/}. É importante depois instalar o
RStudio, disponível em \url{https://www.rstudio.org/}, que é uma IDE
(Integrated Development Enviroment). IDEs são progrmas que facilitam, e
muito, programar porque trazem um ambiente gráfico mais intuitivo,
disponibilizando informações como quais objetos estão carregados na
memória do computador, visualização de gráficos e animações que fazemos
e por aí vai.

\begin{figure}
\includegraphics[width=1\linewidth]{rstudio} \caption{A caption}\label{fig:pressure}
\end{figure}

\hypertarget{links-importantes}{%
\section{Links importantes}\label{links-importantes}}

\begin{itemize}
\tightlist
\item
  \href{https://github.com}{Github}
\end{itemize}

É como uma ``rede social de códigos'' com várias funcionalidades úteis
para cuidar dos seus códigos e gerenciar projetos. É muito importante
fazer uma conta lá e usar o programa para lidar adequadamente com o
armazenamento dos códigos. O Github é particularmente útil para
trabalhar com outras pessoas porque armazena versões antigas, quem fez
que alterações nos códigos, evita conflito entre versões de programas e
mantém tudo numa fonte única e facilmente acessível.

\begin{itemize}
\tightlist
\item
  \href{https://stackoverflow.com/}{StackOverflow}
\end{itemize}

Um fórum extremamente popular em inglês sobre programação. A maior parte
das suas dúvidas já foi resolvida lá e se não foi, é muito simples fazer
uma nova pergunta.

\begin{itemize}
\tightlist
\item
  \href{https://stats.stackexchange.com/}{CrossValidated}
\end{itemize}

Uma espécie de StackOverflow, mas voltada para análise de dados. Quando
a dúvida for mais estatístisca do que de programação em si, é melhor
conferir aqui. A maioria dos usuários sabe R e vai pedir algum pedaço de
código para entender o seu problema bem.

\begin{itemize}
\tightlist
\item
  \href{https://danmrc.github.io/R-para-Economistas/}{R: Uma Introdução
  para Economistas}
\end{itemize}

Uma fonte ótima para consultas rápidas.

\begin{itemize}
\tightlist
\item
  \href{https://r4ds.had.co.nz}{R for Data Science}
\end{itemize}

A fonte definitiva do R introdutório. É um livro em inglês muito extenso
e deve cobrir razoavelmente qualquer assunto que um iniciante queira
entender melhor.

\hypertarget{os-primeiros-comandos}{%
\section{Os primeiros comandos}\label{os-primeiros-comandos}}

Tenha em mente que você pode anotar linhas de código na área do script e
rodar linhas específicas copiando-as no console, digiando-as diretamente
lá ou selecionando o trecho do script desejado e apertando
\texttt{ctrl\ +\ enter}.

A grande vantagem do R é sua natureza colaborativa. Pesquisadores,
programadores, profissionais e entusiastas do mundo todo escrevem
\emph{pacotes} com funcionalidades novas, que incluem coisas como
ferramentas para econometria bayseana, gerar animaçãoes, estimar
dinâmicas evolutivas, resolver problemas de otimização e gerir blogs.
Pacotes tem nomes e eles trazem \emph{funções} novas. Normalmente nos
referimos a uma função na forma \texttt{pacote::função} ou
\texttt{função()}. Então se lemos \texttt{dplyr::filter} sabemos que é a
função \texttt{filter()} do pacote \texttt{dplyr}.

Alguns pacotes já vêm carregados no R, eles são o que chamamos de
Biblioteca Padrão, a versão mais simples do R. Os mais importantes
pacotes da BP são o \texttt{base} com toda a sintaxe básica da
linguagem, o \texttt{stats} com dezenas de ferramentas estatísticas
muito úteis e o \texttt{utils} com várias funções miscelâneas. No
entanto, a maioria dos pacotes não vem carregada no R diretamente. Eles
estão sediados no que chamamos de Comprehensive R Archive Network
(CRAN). Podemos instalar esses pacotes facilmente e depois é simples
carrega-los. Vamos baixar e carregar o \texttt{rootSolve}, que traz
funções para resolver equações e problemas de cálculo diferencial.

\begin{Shaded}
\begin{Highlighting}[]
\KeywordTok{install.packages}\NormalTok{(}\StringTok{"rootSolve"}\NormalTok{)}
\KeywordTok{library}\NormalTok{(rootSolve)}
\end{Highlighting}
\end{Shaded}

\texttt{install.packages()} só precisa que você dê o nome do pacote que
o R instala para você. \texttt{library()} serve para carregar o pacote e
ter essas funções novas disponíveis. Para saber quem fez o pacote e como
devemos cita-lo em trabalhos acadêmicos, basta usar \texttt{citation()}.

\begin{Shaded}
\begin{Highlighting}[]
\KeywordTok{citation}\NormalTok{(}\StringTok{"rootSolve"}\NormalTok{)}
\end{Highlighting}
\end{Shaded}

\begin{verbatim}
## 
## To cite package 'rootSolve' in publications use:
## 
##   Soetaert K. and P.M.J. Herman (2009).  A Practical Guide to
##   Ecological Modelling. Using R as a Simulation Platform.  Springer,
##   372 pp.
## 
##   Soetaert K. (2009).  rootSolve: Nonlinear root finding, equilibrium
##   and steady-state analysis of ordinary differential equations.
##   R-package version 1.6
## 
## rootSolve was created to solve the examples from chapter 7 of our book
## - please cite this book when using it, thank you!
## To see these entries in BibTeX format, use 'print(<citation>,
## bibtex=TRUE)', 'toBibtex(.)', or set
## 'options(citation.bibtex.max=999)'.
\end{verbatim}

Agora vamos cobrir alguns aspectos básicos da sintaxe do R.

\hypertarget{uma-calculadora-potente}{%
\section{Uma calculadora potente}\label{uma-calculadora-potente}}

A maneira mais simples de pensar no R é como uma calculadora. Observe
que podemos usar \texttt{\#} para fazer comentários no código, isso é
útil para deixar tudo mais legível.

\begin{Shaded}
\begin{Highlighting}[]
\DecValTok{2} \OperatorTok{+}\StringTok{ }\DecValTok{2} \CommentTok{# soma simples}
\end{Highlighting}
\end{Shaded}

\begin{verbatim}
## [1] 4
\end{verbatim}

\begin{Shaded}
\begin{Highlighting}[]
\DecValTok{2} \OperatorTok{-}\StringTok{ }\DecValTok{1} \CommentTok{# subtração}
\end{Highlighting}
\end{Shaded}

\begin{verbatim}
## [1] 1
\end{verbatim}

\begin{Shaded}
\begin{Highlighting}[]
\DecValTok{2}\OperatorTok{^}\DecValTok{3} \CommentTok{# elevar ao cubo}
\end{Highlighting}
\end{Shaded}

\begin{verbatim}
## [1] 8
\end{verbatim}

\begin{Shaded}
\begin{Highlighting}[]
\DecValTok{2}\OperatorTok{*}\DecValTok{3} \CommentTok{#multiplicação}
\end{Highlighting}
\end{Shaded}

\begin{verbatim}
## [1] 6
\end{verbatim}

\begin{Shaded}
\begin{Highlighting}[]
\DecValTok{-2}\OperatorTok{*}\DecValTok{3} \CommentTok{# multiplicação por um negativo}
\end{Highlighting}
\end{Shaded}

\begin{verbatim}
## [1] -6
\end{verbatim}

\begin{Shaded}
\begin{Highlighting}[]
\DecValTok{2}\OperatorTok{**}\DecValTok{3} \CommentTok{# forma alternativa de elevar à potências}
\end{Highlighting}
\end{Shaded}

\begin{verbatim}
## [1] 8
\end{verbatim}

Também podemos fazer testes lógicos usando o \texttt{==} para igual ou
\texttt{!=} para diferente.

\begin{Shaded}
\begin{Highlighting}[]
\DecValTok{2} \OperatorTok{+}\StringTok{ }\DecValTok{2} \OperatorTok{==}\StringTok{ }\DecValTok{4} \CommentTok{# testando se 2 + 2 = 4}
\end{Highlighting}
\end{Shaded}

\begin{verbatim}
## [1] TRUE
\end{verbatim}

\begin{Shaded}
\begin{Highlighting}[]
\DecValTok{2} \OperatorTok{+}\StringTok{ }\DecValTok{2} \OperatorTok{!=}\StringTok{ }\DecValTok{4} \CommentTok{# agora se é diferente}
\end{Highlighting}
\end{Shaded}

\begin{verbatim}
## [1] FALSE
\end{verbatim}

\begin{Shaded}
\begin{Highlighting}[]
\DecValTok{2} \OperatorTok{+}\StringTok{ }\DecValTok{2} \OperatorTok{>}\StringTok{ }\DecValTok{3} \CommentTok{# maior}
\end{Highlighting}
\end{Shaded}

\begin{verbatim}
## [1] TRUE
\end{verbatim}

\begin{Shaded}
\begin{Highlighting}[]
\DecValTok{2} \OperatorTok{+}\StringTok{ }\DecValTok{2} \OperatorTok{<}\StringTok{ }\DecValTok{3} \CommentTok{# menor}
\end{Highlighting}
\end{Shaded}

\begin{verbatim}
## [1] FALSE
\end{verbatim}

\begin{Shaded}
\begin{Highlighting}[]
\OtherTok{TRUE} \OperatorTok{==}\StringTok{ }\OtherTok{FALSE} \CommentTok{# podemos também testar proposições lógicas mais abstratas}
\end{Highlighting}
\end{Shaded}

\begin{verbatim}
## [1] FALSE
\end{verbatim}

No entanto, a maior parte do tempo lidaremos com \emph{objetos}, que
iremos definir com o sinal \texttt{\textless{}-}, que chamamos de
Operador de Designação (Assignment Operator). Se parecer muito estranho
digitar isso, o atalho é \texttt{Alt} + \texttt{-}.

\begin{Shaded}
\begin{Highlighting}[]
\NormalTok{a =}\StringTok{ }\DecValTok{2} \CommentTok{# definindo um objeto a como 2}
\NormalTok{b <-}\StringTok{ }\DecValTok{2} \CommentTok{# o mesmo com b, usando o sinal <-}

\NormalTok{a }\OperatorTok{==}\StringTok{ }\NormalTok{b }\CommentTok{# teste lógico}
\end{Highlighting}
\end{Shaded}

\begin{verbatim}
## [1] TRUE
\end{verbatim}

Temos também como usar funções, pedaços prontos de código com
funcionalidades específicas. Funções podem ou não admitir
\emph{argumentos}, que são especificados usando seu nome, um sinal de
\texttt{=} e o valor do argumento, todos separados por vírgula. Por
questões de organização de código, é bom pular uma linha para cada
argumento, embora você possa usar formas alternativas de identação do
texto. Algumas funções são simples o suficiente para que você não
precise dizer exatamente qual argumento é qual, como é o caso de
\texttt{seq()}. \texttt{seq()} também tem outra pecualiaridade, seu
argumento \texttt{by}, que informa o tamanho do passo entre um elemento
e outro da sequência é por padrão o número 1. Descobrimos isso lendo a
documentação da função. Você pode acessa-la pela função \texttt{help()}
ou apertando \texttt{F1} quando o cursor estiver em cima da função.

\begin{Shaded}
\begin{Highlighting}[]
\KeywordTok{help}\NormalTok{(seq)}
\end{Highlighting}
\end{Shaded}

Abaixo listo algumas funções com funcionalidades muito básicas como
\texttt{print()} que retorna no console o valor de algum objeto,
\texttt{exp(x)} que calcula \(e^x\) e \texttt{seq()} para gerar
sequências.

\begin{Shaded}
\begin{Highlighting}[]
\KeywordTok{print}\NormalTok{(a) }\CommentTok{# retorna no console o valor de a}
\end{Highlighting}
\end{Shaded}

\begin{verbatim}
## [1] 2
\end{verbatim}

\begin{Shaded}
\begin{Highlighting}[]
\KeywordTok{exp}\NormalTok{(}\DecValTok{4}\NormalTok{) }\CommentTok{# exponencial}
\end{Highlighting}
\end{Shaded}

\begin{verbatim}
## [1] 54.59815
\end{verbatim}

\begin{Shaded}
\begin{Highlighting}[]
\KeywordTok{factorial}\NormalTok{(}\DecValTok{4}\NormalTok{) }\CommentTok{# fatorial}
\end{Highlighting}
\end{Shaded}

\begin{verbatim}
## [1] 24
\end{verbatim}

\begin{Shaded}
\begin{Highlighting}[]
\KeywordTok{sqrt}\NormalTok{(}\DecValTok{9}\NormalTok{) }\CommentTok{# raiz quadrada}
\end{Highlighting}
\end{Shaded}

\begin{verbatim}
## [1] 3
\end{verbatim}

\begin{Shaded}
\begin{Highlighting}[]
\KeywordTok{choose}\NormalTok{(}\DecValTok{4}\NormalTok{, }\DecValTok{2}\NormalTok{) }\CommentTok{# permutação de 4, 2 a 2}
\end{Highlighting}
\end{Shaded}

\begin{verbatim}
## [1] 6
\end{verbatim}

\begin{Shaded}
\begin{Highlighting}[]
\KeywordTok{seq}\NormalTok{(}\DataTypeTok{from =} \DecValTok{1}\NormalTok{,}
    \DataTypeTok{to =} \DecValTok{10}\NormalTok{, }
    \DataTypeTok{by =} \DecValTok{2}\NormalTok{) }\CommentTok{#sequência de-para com passo 2}
\end{Highlighting}
\end{Shaded}

\begin{verbatim}
## [1] 1 3 5 7 9
\end{verbatim}

\begin{Shaded}
\begin{Highlighting}[]
\KeywordTok{seq}\NormalTok{(}\DecValTok{1}\NormalTok{, }\DecValTok{10}\NormalTok{, }\DecValTok{2}\NormalTok{) }\CommentTok{# o mesmo resultado sem especificar qual argumento é qual}
\end{Highlighting}
\end{Shaded}

\begin{verbatim}
## [1] 1 3 5 7 9
\end{verbatim}

\begin{Shaded}
\begin{Highlighting}[]
\NormalTok{c <-}\StringTok{ }\KeywordTok{seq}\NormalTok{(}\DecValTok{1}\NormalTok{, }\DecValTok{5}\NormalTok{, }\DecValTok{1}\NormalTok{) }\CommentTok{# agora com um passo 1}

\DecValTok{1}\OperatorTok{:}\DecValTok{5} \CommentTok{# usar : também serve para gerar sequências com passo 1}
\end{Highlighting}
\end{Shaded}

\begin{verbatim}
## [1] 1 2 3 4 5
\end{verbatim}

\begin{Shaded}
\begin{Highlighting}[]
\KeywordTok{print}\NormalTok{(c)}
\end{Highlighting}
\end{Shaded}

\begin{verbatim}
## [1] 1 2 3 4 5
\end{verbatim}

\begin{Shaded}
\begin{Highlighting}[]
\KeywordTok{sum}\NormalTok{(c) }\CommentTok{# soma das entradas}
\end{Highlighting}
\end{Shaded}

\begin{verbatim}
## [1] 15
\end{verbatim}

\begin{Shaded}
\begin{Highlighting}[]
\KeywordTok{mean}\NormalTok{(c) }\CommentTok{# média de c}
\end{Highlighting}
\end{Shaded}

\begin{verbatim}
## [1] 3
\end{verbatim}

Observe que o objeto \texttt{c} é um pouco diferente dos anteriores, que
eram só um número. \texttt{c} tem uma sequência. Para descobrir a classe
de um objeto, usamos a função \texttt{class()} e para inspeciona-lo
melhor usar \texttt{str()} (uma abreviação de \emph{estrutura} em
inglês).

\begin{Shaded}
\begin{Highlighting}[]
\KeywordTok{class}\NormalTok{(a)}
\end{Highlighting}
\end{Shaded}

\begin{verbatim}
## [1] "numeric"
\end{verbatim}

\begin{Shaded}
\begin{Highlighting}[]
\KeywordTok{str}\NormalTok{(a)}
\end{Highlighting}
\end{Shaded}

\begin{verbatim}
##  num 2
\end{verbatim}

\begin{Shaded}
\begin{Highlighting}[]
\KeywordTok{class}\NormalTok{(c)}
\end{Highlighting}
\end{Shaded}

\begin{verbatim}
## [1] "numeric"
\end{verbatim}

\begin{Shaded}
\begin{Highlighting}[]
\KeywordTok{str}\NormalTok{(c)}
\end{Highlighting}
\end{Shaded}

\begin{verbatim}
##  num [1:5] 1 2 3 4 5
\end{verbatim}

Uma das estruturas de dados mais importantes do R são \emph{vetores}.
Podemos declarar vetores de várias formas, uma ``simples'' é usando a
função \texttt{c()}. Para saber se um objeto é vetor, podemos usar
\texttt{is.vector()}. Objetos podem ser nomeados com números, desde que
não comecem com um número. Letras maísculas e minúsculas são
diferenciadas, então tenha isso em mente ao nomear objetos.

\begin{Shaded}
\begin{Highlighting}[]
\NormalTok{A <-}\StringTok{ }\KeywordTok{c}\NormalTok{(}\DecValTok{2}\NormalTok{,}\DecValTok{2}\NormalTok{) }\CommentTok{# A é um vetor de duas dimensões em que cada entrada é um 2}

\KeywordTok{print}\NormalTok{(A) }\CommentTok{# printamos no console}
\end{Highlighting}
\end{Shaded}

\begin{verbatim}
## [1] 2 2
\end{verbatim}

\begin{Shaded}
\begin{Highlighting}[]
\KeywordTok{class}\NormalTok{(A) }\CommentTok{# descobrimos a classe}
\end{Highlighting}
\end{Shaded}

\begin{verbatim}
## [1] "numeric"
\end{verbatim}

\begin{Shaded}
\begin{Highlighting}[]
\KeywordTok{str}\NormalTok{(A) }\CommentTok{# inspecionamos a estrutura}
\end{Highlighting}
\end{Shaded}

\begin{verbatim}
##  num [1:2] 2 2
\end{verbatim}

\begin{Shaded}
\begin{Highlighting}[]
\KeywordTok{is.vector}\NormalTok{(A)}
\end{Highlighting}
\end{Shaded}

\begin{verbatim}
## [1] TRUE
\end{verbatim}

Um truque para se poupar de escrever muitos prints se precisar é
escrever a linha de código toda entre parênteses. Como por exemplo:

\begin{Shaded}
\begin{Highlighting}[]
\NormalTok{(B <-}\StringTok{ }\KeywordTok{c}\NormalTok{(}\DecValTok{2}\NormalTok{, }\DecValTok{-3}\NormalTok{, }\DecValTok{5}\NormalTok{, }\DecValTok{8}\NormalTok{))}
\end{Highlighting}
\end{Shaded}

\begin{verbatim}
## [1]  2 -3  5  8
\end{verbatim}

\begin{Shaded}
\begin{Highlighting}[]
\NormalTok{C <-}\StringTok{ }\KeywordTok{c}\NormalTok{(}\StringTok{"um"}\NormalTok{, }\StringTok{"dois"}\NormalTok{, }\StringTok{"madeira"}\NormalTok{, }\StringTok{"peixe"}\NormalTok{, }\StringTok{"PET-UFF"}\NormalTok{, }\StringTok{"Niterói")}

\StringTok{D <- c(TRUE, FALSE, TRUE, FALSE, FALSE)}
\end{Highlighting}
\end{Shaded}

Vale parar brevemente para falar de fatores. Até agora trabalhamos com
dados lógicos ou numéricos, mas é muito comum encontrar dados
categóricos como por exemplo sexo ou profissão. É mais simples lidar com
esse tipo de dado quando é declarado como um fator. Isso é simples,
basta usar a função \texttt{factor()}. Esse tipo é muito útil para rodar
modelos com variáveis categóricas porque o R faz por nós o trabalho de
criar variáveis dummy com cada classe e nos informa quais tipos foram
observados. Para lidar com fatores também é comum usar
\texttt{levels()}, que retorna os tipos encontrados no vetor. Digamos
por exemplo que \texttt{C2} seja uma lista de extensções de que
participam 10 alunos aleatoriamente escolhidos.

\begin{Shaded}
\begin{Highlighting}[]
\NormalTok{C2 <-}\StringTok{ }\KeywordTok{c}\NormalTok{(}\StringTok{"PET"}\NormalTok{, }\StringTok{"Atlética"}\NormalTok{, }\StringTok{"PET"}\NormalTok{, }\StringTok{"Goal"}\NormalTok{, }\StringTok{"Opção"}\NormalTok{, }\StringTok{"PET"}\NormalTok{, }\StringTok{"Opção"}\NormalTok{, }\StringTok{"Atlética"}\NormalTok{, }\StringTok{"Opção"}\NormalTok{, }\StringTok{"PET"}\NormalTok{)}
\NormalTok{C2 <-}\StringTok{ }\KeywordTok{factor}\NormalTok{(C2)}

\NormalTok{C2}
\end{Highlighting}
\end{Shaded}

\begin{verbatim}
##  [1] PET      Atlética PET      Goal     Opção    PET      Opção    Atlética
##  [9] Opção    PET     
## Levels: Atlética Goal Opção PET
\end{verbatim}

\begin{Shaded}
\begin{Highlighting}[]
\KeywordTok{levels}\NormalTok{(C2)}
\end{Highlighting}
\end{Shaded}

\begin{verbatim}
## [1] "Atlética" "Goal"     "Opção"    "PET"
\end{verbatim}

Operações com vetores são bem intuitivas no R porque a maioria das
funções é vetorizada - são aplicadas à todos os elementos de um vetor. A
função \texttt{ifelse()} por exemplo - que deve lembrar aos usuários de
Excell a função \texttt{SE} - também funciona no mesmo espírito. Basta
especificarmos um teste lógico, uma resposta para verdadeiro e outra
para falso.

\begin{Shaded}
\begin{Highlighting}[]
\NormalTok{E <-}\StringTok{ }\KeywordTok{c}\NormalTok{(}\DecValTok{1}\NormalTok{, }\DecValTok{3}\NormalTok{, }\DecValTok{4}\NormalTok{, }\DecValTok{9}\NormalTok{)}

\NormalTok{F_ =}\StringTok{ }\NormalTok{B}\OperatorTok{*}\NormalTok{E }\CommentTok{# multiplicação de vetores}

\CommentTok{# nunca declare um objeto chamado F ou T porque são os símbolos de verdadeiro e falso}

\NormalTok{B}\OperatorTok{*}\NormalTok{E }\CommentTok{# podemos também somente recuperar a conta se não quisermos printar F_}
\end{Highlighting}
\end{Shaded}

\begin{verbatim}
## [1]  2 -9 20 72
\end{verbatim}

\begin{Shaded}
\begin{Highlighting}[]
\NormalTok{G <-}\StringTok{ }\KeywordTok{ifelse}\NormalTok{(C }\OperatorTok{==}\StringTok{ "PET-UFF"}\NormalTok{, }\CommentTok{# testa se cada entrada é igual a "PET-UFF"}
           \StringTok{"Entrada do PET"}\NormalTok{, }\CommentTok{# resposta se for}
           \StringTok{"Não é a Entrada do PET"}\NormalTok{) }\CommentTok{# resposta se não for}
\KeywordTok{print}\NormalTok{(G)}
\end{Highlighting}
\end{Shaded}

\begin{verbatim}
## [1] "Não é a Entrada do PET" "Não é a Entrada do PET" "Não é a Entrada do PET"
## [4] "Não é a Entrada do PET" "Entrada do PET"         "Não é a Entrada do PET"
\end{verbatim}

O próximo objeto são matrizes. Matrizes precisam de vetores do mesmo
tipo para funcionar. Precisamos alimentar um vetor só e depois
especificar quantas linhas e colunas queremos. Podemos pedir os
autovalores e autovetores da matriz e também podemos multiplicar
matrizes com \texttt{\%*\%}.

\begin{Shaded}
\begin{Highlighting}[]
\NormalTok{H <-}\StringTok{ }\KeywordTok{c}\NormalTok{(}\DecValTok{1}\NormalTok{, }\DecValTok{3}\NormalTok{, }\DecValTok{2}\NormalTok{, }\DecValTok{4}\NormalTok{) }

\NormalTok{I <-}\StringTok{ }\KeywordTok{matrix}\NormalTok{(H,        }\CommentTok{# vetor}
           \DataTypeTok{nrow =} \DecValTok{2}\NormalTok{, }\CommentTok{# número de linhas}
           \DataTypeTok{ncol =} \DecValTok{2}\NormalTok{) }\CommentTok{# número de colunas}

\NormalTok{auto <-}\StringTok{ }\KeywordTok{eigen}\NormalTok{(I) }\CommentTok{#autovalores e autovetores da matriz}
\KeywordTok{print}\NormalTok{(auto)}
\end{Highlighting}
\end{Shaded}

\begin{verbatim}
## eigen() decomposition
## $values
## [1]  5.3722813 -0.3722813
## 
## $vectors
##            [,1]       [,2]
## [1,] -0.4159736 -0.8245648
## [2,] -0.9093767  0.5657675
\end{verbatim}

\begin{Shaded}
\begin{Highlighting}[]
\NormalTok{J <-}\StringTok{ }\KeywordTok{c}\NormalTok{(}\DecValTok{2}\NormalTok{, }\DecValTok{1}\NormalTok{, }\DecValTok{5}\NormalTok{, }\DecValTok{3}\NormalTok{)}

\NormalTok{K <-}\StringTok{ }\KeywordTok{matrix}\NormalTok{(J,}
           \DataTypeTok{ncol =} \DecValTok{2}\NormalTok{)}

\NormalTok{I }\OperatorTok\StringTok{ }\NormalTok{K }\CommentTok{## multiplicação de matrizes}
\end{Highlighting}
\end{Shaded}

\begin{verbatim}
##      [,1] [,2]
## [1,]    4   11
## [2,]   10   27
\end{verbatim}

\hypertarget{exercuxedcios}{%
\section{Exercícios}\label{exercuxedcios}}

\begin{itemize}
\tightlist
\item
  Calcule \(52e^2 - 35 \times 4!\)
\item
  Ache, se existirem, os autovetores da matriz: \[A = \begin{bmatrix}
  1&2&5\\
  0&3&1\\
  2&4&3\\
  \end{bmatrix}\]
\item
  Ache a transposta de \(A\)
\item
  Calcule a soma dos autovalores da matriz \(A\).
\item
  Calcule a média de uma sequência começando em \(150\), terminando em
  \(500\), com passo \(0,5\)
\end{itemize}

\hypertarget{data-frames}{%
\section{Data Frames}\label{data-frames}}

A estrutura de dados mais comum no dia a dia da análise de dados é um
Data Frame. Usamos a função \texttt{data.frame()} para gera-los. Um DF é
uma coleção de vetores que admite tipos \emph{diferentes}, então são
mais flexíveis que matrizes. Na hora de declarar o DF, podemos dar nome
a cada vetor. Observe que precisamos que todos os vetores do DF tenham o
\emph{mesmo} comprimento. Podemos usar a função \texttt{length()} para
averiguar isso.

\begin{Shaded}
\begin{Highlighting}[]
\NormalTok{elemento1 <-}\StringTok{ }\KeywordTok{seq}\NormalTok{(}\DecValTok{1}\NormalTok{, }\DecValTok{100}\NormalTok{)}
\NormalTok{elemento2 <-}\StringTok{ }\KeywordTok{seq}\NormalTok{(}\DecValTok{50}\NormalTok{, }\DecValTok{150}\NormalTok{)}

\KeywordTok{length}\NormalTok{(elemento1)}
\end{Highlighting}
\end{Shaded}

\begin{verbatim}
## [1] 100
\end{verbatim}

\begin{Shaded}
\begin{Highlighting}[]
\KeywordTok{length}\NormalTok{(elemento2)}
\end{Highlighting}
\end{Shaded}

\begin{verbatim}
## [1] 101
\end{verbatim}

\begin{Shaded}
\begin{Highlighting}[]
\NormalTok{base <-}\StringTok{ }\KeywordTok{data.frame}\NormalTok{(}\DataTypeTok{primeiro =}\NormalTok{ elemento1,}
                  \DataTypeTok{segundo =}\NormalTok{ elemento2)}
\end{Highlighting}
\end{Shaded}

\begin{verbatim}
## Error in data.frame(primeiro = elemento1, segundo = elemento2): arguments imply differing number of rows: 100, 101
\end{verbatim}

No entanto se fizermos \texttt{elemento1} e \texttt{elemento2} terem o
mesmo comprimento, o DF sai sem problemas. A função \texttt{head()} puxa
o cabeçalho de um objeto e é muito útil para averiguar visualmente se
está tudo como esperado.

\begin{Shaded}
\begin{Highlighting}[]
\NormalTok{elemento1 <-}\StringTok{ }\KeywordTok{seq}\NormalTok{(}\DecValTok{1}\NormalTok{, }\DecValTok{100}\NormalTok{)}
\NormalTok{elemento2 <-}\StringTok{ }\KeywordTok{seq}\NormalTok{(}\DecValTok{50}\NormalTok{, }\DecValTok{149}\NormalTok{)}

\KeywordTok{length}\NormalTok{(elemento1)}
\end{Highlighting}
\end{Shaded}

\begin{verbatim}
## [1] 100
\end{verbatim}

\begin{Shaded}
\begin{Highlighting}[]
\KeywordTok{length}\NormalTok{(elemento2)}
\end{Highlighting}
\end{Shaded}

\begin{verbatim}
## [1] 100
\end{verbatim}

\begin{Shaded}
\begin{Highlighting}[]
\NormalTok{base <-}\StringTok{ }\KeywordTok{data.frame}\NormalTok{(}\DataTypeTok{primeiro =}\NormalTok{ elemento1,}
                  \DataTypeTok{segundo =}\NormalTok{ elemento2)}

\KeywordTok{head}\NormalTok{(base)}
\end{Highlighting}
\end{Shaded}

\begin{verbatim}
##   primeiro segundo
## 1        1      50
## 2        2      51
## 3        3      52
## 4        4      53
## 5        5      54
## 6        6      55
\end{verbatim}

Nos referimos aos elementos de um DF pelo símbolo \texttt{\$}. Então se
quisermos resgatar somente o vetor \texttt{primeiro}, usamos
\texttt{base\$primeiro}. Isso também vale se quisermos criar mais
vetores na base. Também podemos nos referir a elementos específicos
usando chaves.

\begin{Shaded}
\begin{Highlighting}[]
\NormalTok{base}\OperatorTok{$}\NormalTok{terceiro <-}\StringTok{ }\NormalTok{base}\OperatorTok{$}\NormalTok{primeiro }\OperatorTok{+}\StringTok{ }\NormalTok{base}\OperatorTok{$}\NormalTok{segundo }

\KeywordTok{mean}\NormalTok{(base}\OperatorTok{$}\NormalTok{terceiro) }\CommentTok{# média}
\end{Highlighting}
\end{Shaded}

\begin{verbatim}
## [1] 150
\end{verbatim}

\begin{Shaded}
\begin{Highlighting}[]
\KeywordTok{median}\NormalTok{(base}\OperatorTok{$}\NormalTok{primeiro) }\CommentTok{# mediana}
\end{Highlighting}
\end{Shaded}

\begin{verbatim}
## [1] 50.5
\end{verbatim}

\begin{Shaded}
\begin{Highlighting}[]
\KeywordTok{summary}\NormalTok{(base}\OperatorTok{$}\NormalTok{segundo) }\CommentTok{# sumário estatístico}
\end{Highlighting}
\end{Shaded}

\begin{verbatim}
##    Min. 1st Qu.  Median    Mean 3rd Qu.    Max. 
##   50.00   74.75   99.50   99.50  124.25  149.00
\end{verbatim}

\begin{Shaded}
\begin{Highlighting}[]
\KeywordTok{rowMeans}\NormalTok{(base) }\CommentTok{# média de cada linha}
\end{Highlighting}
\end{Shaded}

\begin{verbatim}
##   [1]  34.00000  35.33333  36.66667  38.00000  39.33333  40.66667  42.00000
##   [8]  43.33333  44.66667  46.00000  47.33333  48.66667  50.00000  51.33333
##  [15]  52.66667  54.00000  55.33333  56.66667  58.00000  59.33333  60.66667
##  [22]  62.00000  63.33333  64.66667  66.00000  67.33333  68.66667  70.00000
##  [29]  71.33333  72.66667  74.00000  75.33333  76.66667  78.00000  79.33333
##  [36]  80.66667  82.00000  83.33333  84.66667  86.00000  87.33333  88.66667
##  [43]  90.00000  91.33333  92.66667  94.00000  95.33333  96.66667  98.00000
##  [50]  99.33333 100.66667 102.00000 103.33333 104.66667 106.00000 107.33333
##  [57] 108.66667 110.00000 111.33333 112.66667 114.00000 115.33333 116.66667
##  [64] 118.00000 119.33333 120.66667 122.00000 123.33333 124.66667 126.00000
##  [71] 127.33333 128.66667 130.00000 131.33333 132.66667 134.00000 135.33333
##  [78] 136.66667 138.00000 139.33333 140.66667 142.00000 143.33333 144.66667
##  [85] 146.00000 147.33333 148.66667 150.00000 151.33333 152.66667 154.00000
##  [92] 155.33333 156.66667 158.00000 159.33333 160.66667 162.00000 163.33333
##  [99] 164.66667 166.00000
\end{verbatim}

\begin{Shaded}
\begin{Highlighting}[]
\KeywordTok{colMeans}\NormalTok{(base) }\CommentTok{# média de cada coluna}
\end{Highlighting}
\end{Shaded}

\begin{verbatim}
## primeiro  segundo terceiro 
##     50.5     99.5    150.0
\end{verbatim}

\begin{Shaded}
\begin{Highlighting}[]
\NormalTok{base[}\DecValTok{1}\NormalTok{,}\DecValTok{2}\NormalTok{] }\CommentTok{# pega o elemento na primeira linha e segunda coluna}
\end{Highlighting}
\end{Shaded}

\begin{verbatim}
## [1] 50
\end{verbatim}

\begin{Shaded}
\begin{Highlighting}[]
\NormalTok{base[}\DecValTok{1}\NormalTok{,] }\CommentTok{# pega a primeira linha}
\end{Highlighting}
\end{Shaded}

\begin{verbatim}
##   primeiro segundo terceiro
## 1        1      50       51
\end{verbatim}

\begin{Shaded}
\begin{Highlighting}[]
\NormalTok{base[base}\OperatorTok{$}\NormalTok{terceiro }\OperatorTok{>}\StringTok{ }\DecValTok{30}\NormalTok{,] }\CommentTok{# pega só as linhas em que a variável terceiro é maior que 30, vale para outros testes lógicos }
\end{Highlighting}
\end{Shaded}

\begin{verbatim}
##     primeiro segundo terceiro
## 1          1      50       51
## 2          2      51       53
## 3          3      52       55
## 4          4      53       57
## 5          5      54       59
## 6          6      55       61
## 7          7      56       63
## 8          8      57       65
## 9          9      58       67
## 10        10      59       69
## 11        11      60       71
## 12        12      61       73
## 13        13      62       75
## 14        14      63       77
## 15        15      64       79
## 16        16      65       81
## 17        17      66       83
## 18        18      67       85
## 19        19      68       87
## 20        20      69       89
## 21        21      70       91
## 22        22      71       93
## 23        23      72       95
## 24        24      73       97
## 25        25      74       99
## 26        26      75      101
## 27        27      76      103
## 28        28      77      105
## 29        29      78      107
## 30        30      79      109
## 31        31      80      111
## 32        32      81      113
## 33        33      82      115
## 34        34      83      117
## 35        35      84      119
## 36        36      85      121
## 37        37      86      123
## 38        38      87      125
## 39        39      88      127
## 40        40      89      129
## 41        41      90      131
## 42        42      91      133
## 43        43      92      135
## 44        44      93      137
## 45        45      94      139
## 46        46      95      141
## 47        47      96      143
## 48        48      97      145
## 49        49      98      147
## 50        50      99      149
## 51        51     100      151
## 52        52     101      153
## 53        53     102      155
## 54        54     103      157
## 55        55     104      159
## 56        56     105      161
## 57        57     106      163
## 58        58     107      165
## 59        59     108      167
## 60        60     109      169
## 61        61     110      171
## 62        62     111      173
## 63        63     112      175
## 64        64     113      177
## 65        65     114      179
## 66        66     115      181
## 67        67     116      183
## 68        68     117      185
## 69        69     118      187
## 70        70     119      189
## 71        71     120      191
## 72        72     121      193
## 73        73     122      195
## 74        74     123      197
## 75        75     124      199
## 76        76     125      201
## 77        77     126      203
## 78        78     127      205
## 79        79     128      207
## 80        80     129      209
## 81        81     130      211
## 82        82     131      213
## 83        83     132      215
## 84        84     133      217
## 85        85     134      219
## 86        86     135      221
## 87        87     136      223
## 88        88     137      225
## 89        89     138      227
## 90        90     139      229
## 91        91     140      231
## 92        92     141      233
## 93        93     142      235
## 94        94     143      237
## 95        95     144      239
## 96        96     145      241
## 97        97     146      243
## 98        98     147      245
## 99        99     148      247
## 100      100     149      249
\end{verbatim}

E também temos listas. São formas bem gerais de estruturas de dados,
porque adimitem qualquer coisa. O primeiro elemento de uma lista pode
ser um DF, o segundo um vetor e o terceiro uma letra.

\begin{Shaded}
\begin{Highlighting}[]
\NormalTok{lista <-}\StringTok{ }\KeywordTok{list}\NormalTok{(}\DataTypeTok{primeiro =}\NormalTok{ base,}
             \DataTypeTok{segundo =} \KeywordTok{seq}\NormalTok{(}\DecValTok{1}\NormalTok{, }\DecValTok{10}\NormalTok{),}
             \DataTypeTok{terceiro =} \StringTok{"a"}\NormalTok{)}

\KeywordTok{class}\NormalTok{(lista)}
\end{Highlighting}
\end{Shaded}

\begin{verbatim}
## [1] "list"
\end{verbatim}

\begin{Shaded}
\begin{Highlighting}[]
\KeywordTok{str}\NormalTok{(lista)}
\end{Highlighting}
\end{Shaded}

\begin{verbatim}
## List of 3
##  $ primeiro:'data.frame':    100 obs. of  3 variables:
##   ..$ primeiro: int [1:100] 1 2 3 4 5 6 7 8 9 10 ...
##   ..$ segundo : int [1:100] 50 51 52 53 54 55 56 57 58 59 ...
##   ..$ terceiro: int [1:100] 51 53 55 57 59 61 63 65 67 69 ...
##  $ segundo : int [1:10] 1 2 3 4 5 6 7 8 9 10
##  $ terceiro: chr "a"
\end{verbatim}

\begin{Shaded}
\begin{Highlighting}[]
\NormalTok{lista}\OperatorTok{$}\NormalTok{terceiro}
\end{Highlighting}
\end{Shaded}

\begin{verbatim}
## [1] "a"
\end{verbatim}

Lembra quando tiramos os autovalores de uma matriz? Salvamos eles no
objeto \texttt{auto}. Bem, como várias funções, \texttt{eigen()} retorna
uma lista com elementos. Em particular, \texttt{eigen()} retorna um tipo
particular de lista chamado \emph{eigen} em que a primeira entrada é um
vetor com os autovalores e a segunda é uma matriz com os autovetores.

\begin{Shaded}
\begin{Highlighting}[]
\NormalTok{auto}\OperatorTok{$}\NormalTok{values}
\end{Highlighting}
\end{Shaded}

\begin{verbatim}
## [1]  5.3722813 -0.3722813
\end{verbatim}

\begin{Shaded}
\begin{Highlighting}[]
\NormalTok{auto}\OperatorTok{$}\NormalTok{vectors}
\end{Highlighting}
\end{Shaded}

\begin{verbatim}
##            [,1]       [,2]
## [1,] -0.4159736 -0.8245648
## [2,] -0.9093767  0.5657675
\end{verbatim}

Antes de prosseguir para ambientes controlados, vamos parar para falar
brevemente de Pastas de Trabalho. Elas são importantes porque vão
facilitar demais a sua vida. Sempre que você precisar que o R leia um
arquivo fora da pasta de trabalho, vai precisar dar o endereço
\emph{completo} dele, o que é chato, apesar de simples. Para definir um
endereço de trabalho, é só coloca-lo entre aspas na função
\texttt{setwd()}. Para descobrir qual é o endereço atual, basta usar
\texttt{getwd()} sem argumentos.

\hypertarget{ambientes-controlados}{%
\section{Ambientes controlados}\label{ambientes-controlados}}

Ambientes controlados são maneiras de organizar testes lógicos e ações a
serem tomadas para resultados diferentes. Vamos cobrir as duas funções
mais comuns, o loop \texttt{for()} e o ambiente \texttt{if()} e expandir
um pouco nossa capacidade de fazer testes lógicos com os operadores E e
OU.

\texttt{2+2\ ==\ 4} sempre irá retornar um verdadeiro, mas se testarmos
se esse enunciado é verdadeiro conjuntamente com outro as garantias vão
embora. Para juntar enunciados lógicos no R usando o conectivo E usamos
a letra \texttt{\&}.

\begin{Shaded}
\begin{Highlighting}[]
\DecValTok{2} \OperatorTok{+}\StringTok{ }\DecValTok{2} \OperatorTok{==}\StringTok{ }\DecValTok{4}
\end{Highlighting}
\end{Shaded}

\begin{verbatim}
## [1] TRUE
\end{verbatim}

\begin{Shaded}
\begin{Highlighting}[]
\DecValTok{2} \OperatorTok{+}\StringTok{ }\DecValTok{2} \OperatorTok{==}\StringTok{ }\DecValTok{4} \OperatorTok{&}\StringTok{ }\DecValTok{3} \OperatorTok{+}\StringTok{ }\DecValTok{3} \OperatorTok{==}\StringTok{ }\DecValTok{6} \CommentTok{# uma verdadeira & uma verdadeira}
\end{Highlighting}
\end{Shaded}

\begin{verbatim}
## [1] TRUE
\end{verbatim}

\begin{Shaded}
\begin{Highlighting}[]
\DecValTok{2} \OperatorTok{+}\StringTok{ }\DecValTok{2} \OperatorTok{==}\StringTok{ }\DecValTok{4} \OperatorTok{&}\StringTok{ }\DecValTok{3} \OperatorTok{-}\StringTok{ }\DecValTok{2} \OperatorTok{==}\StringTok{ }\DecValTok{-1} \CommentTok{# uma verdadeira & uma falsa}
\end{Highlighting}
\end{Shaded}

\begin{verbatim}
## [1] FALSE
\end{verbatim}

Também podemos testar se uma ou outra são verdadeiras, nesse caso usamos
a barra vertical \texttt{\textbar{}}, acionada com
\texttt{Shift\ +\ \textbackslash{}}.

\begin{Shaded}
\begin{Highlighting}[]
\DecValTok{2} \OperatorTok{+}\StringTok{ }\DecValTok{2} \OperatorTok{==}\StringTok{ }\DecValTok{4} \OperatorTok{|}\StringTok{ }\DecValTok{3} \OperatorTok{+}\StringTok{ }\DecValTok{3} \OperatorTok{==}\StringTok{ }\DecValTok{6} \CommentTok{# uma verdadeira ou uma verdadeira}
\end{Highlighting}
\end{Shaded}

\begin{verbatim}
## [1] TRUE
\end{verbatim}

\begin{Shaded}
\begin{Highlighting}[]
\DecValTok{2} \OperatorTok{+}\StringTok{ }\DecValTok{2} \OperatorTok{==}\StringTok{ }\DecValTok{4} \OperatorTok{|}\StringTok{ }\DecValTok{3} \OperatorTok{-}\StringTok{ }\DecValTok{2} \OperatorTok{==}\StringTok{ }\DecValTok{-1} \CommentTok{# uma verdadeira ou uma falsa}
\end{Highlighting}
\end{Shaded}

\begin{verbatim}
## [1] TRUE
\end{verbatim}

\hypertarget{o-enunciado-if}{%
\subsection{\texorpdfstring{O enunciado
\texttt{if}}{O enunciado if}}\label{o-enunciado-if}}

O enunciado \texttt{if} segue sempre a mesma estrutura:

\begin{Shaded}
\begin{Highlighting}[]
\ControlFlowTok{if}\NormalTok{(Condição }\OperatorTok{==}\StringTok{ }\NormalTok{Verdadeira) \{}
  
\NormalTok{  Expressão}
  
\NormalTok{  \}}
\end{Highlighting}
\end{Shaded}

Digamos que temos um vetor com dados de vendas mensais e uma meta,
poderíamos então fazer:

\begin{Shaded}
\begin{Highlighting}[]
\NormalTok{meta <-}\StringTok{ }\DecValTok{200}
\NormalTok{vendas_mensais <-}\StringTok{ }\KeywordTok{c}\NormalTok{(}\DecValTok{12}\NormalTok{, }\DecValTok{15}\NormalTok{, }\DecValTok{18}\NormalTok{, }\DecValTok{25}\NormalTok{, }\DecValTok{30}\NormalTok{, }\DecValTok{16}\NormalTok{, }\DecValTok{20}\NormalTok{, }\DecValTok{12}\NormalTok{, }\DecValTok{13}\NormalTok{, }\DecValTok{15}\NormalTok{, }\DecValTok{16}\NormalTok{, }\DecValTok{22}\NormalTok{)}

\ControlFlowTok{if}\NormalTok{(}\KeywordTok{sum}\NormalTok{(vendas_mensais) }\OperatorTok{>}\StringTok{ }\NormalTok{meta) \{}
  
  \KeywordTok{print}\NormalTok{(}\StringTok{"Meta de vendas cumprida"}\NormalTok{)}

\NormalTok{  \}}
\end{Highlighting}
\end{Shaded}

\begin{verbatim}
## [1] "Meta de vendas cumprida"
\end{verbatim}

Podemos rebuscar um pouco isso usando funções como \texttt{paste()} que
agrupa pedaços de texto e \texttt{else} para definir o que deve ser
feito caso o teste lógico retorne Falso.

\begin{Shaded}
\begin{Highlighting}[]
\ControlFlowTok{if}\NormalTok{(}\KeywordTok{sum}\NormalTok{(vendas_mensais) }\OperatorTok{>}\StringTok{ }\NormalTok{meta) \{}
  
\NormalTok{  diferenca <-}\StringTok{ }\KeywordTok{sum}\NormalTok{(vendas_mensais) }\OperatorTok{-}\StringTok{ }\NormalTok{meta}
  
  \KeywordTok{print}\NormalTok{(}\KeywordTok{paste}\NormalTok{(}\StringTok{"Meta de vendas cumprida com margem de"}\NormalTok{, diferenca))}

\NormalTok{  \} }
\end{Highlighting}
\end{Shaded}

\begin{verbatim}
## [1] "Meta de vendas cumprida com margem de 14"
\end{verbatim}

Agora com um \texttt{else} a estrutura é essencialmente a mesma:

\begin{Shaded}
\begin{Highlighting}[]
\ControlFlowTok{if}\NormalTok{(Condição }\OperatorTok{==}\StringTok{ }\NormalTok{Verdadeira) \{}
  
\NormalTok{  Expressão}
  
\NormalTok{\} }\ControlFlowTok{else}\NormalTok{ \{}
    
\NormalTok{  Expressão alternativa}
  
\NormalTok{  \}}
\end{Highlighting}
\end{Shaded}

Como por exemplo:

\begin{Shaded}
\begin{Highlighting}[]
\ControlFlowTok{if}\NormalTok{(}\KeywordTok{sum}\NormalTok{(vendas_mensais) }\OperatorTok{>}\StringTok{ }\NormalTok{meta) \{}
  
\NormalTok{  diferenca =}\StringTok{ }\KeywordTok{sum}\NormalTok{(vendas_mensais) }\OperatorTok{-}\StringTok{ }\NormalTok{meta}
  
  \KeywordTok{print}\NormalTok{(}\KeywordTok{paste}\NormalTok{(}\StringTok{"Meta de vendas cumprida com margem de"}\NormalTok{, diferenca))}

\NormalTok{\} }\ControlFlowTok{else}\NormalTok{ \{}
  
\NormalTok{  diferenca =}\StringTok{ }\KeywordTok{sum}\NormalTok{(vendas_mensais) }\OperatorTok{-}\StringTok{ }\NormalTok{meta}
  
  \KeywordTok{print}\NormalTok{(}\KeywordTok{paste}\NormalTok{(}\StringTok{"Meta de vendas não foi cumprida, com diferença de"}\NormalTok{, diferenca))  }
  
\NormalTok{  \}}
\end{Highlighting}
\end{Shaded}

\begin{verbatim}
## [1] "Meta de vendas cumprida com margem de 14"
\end{verbatim}

Observe que repetimos o cálculo da diferença dentro de cada opção. Isso
é importante porque o que quer que esteja dentro das chaves só vai ser
executado se o teste lógico retornar um resultado específico. Poderíamos
definir também a diferença do lado de fora do \texttt{if()} para não
precisarmos repetir.

\begin{Shaded}
\begin{Highlighting}[]
\NormalTok{diferenca =}\StringTok{ }\KeywordTok{sum}\NormalTok{(vendas_mensais) }\OperatorTok{-}\StringTok{ }\NormalTok{meta}

\ControlFlowTok{if}\NormalTok{(}\KeywordTok{sum}\NormalTok{(vendas_mensais) }\OperatorTok{>}\StringTok{ }\NormalTok{meta) \{}
  
  \KeywordTok{print}\NormalTok{(}\KeywordTok{paste}\NormalTok{(}\StringTok{"Meta de vendas cumprida com margem de"}\NormalTok{, diferenca))}

\NormalTok{\} }\ControlFlowTok{else}\NormalTok{ \{}
  
  \KeywordTok{print}\NormalTok{(}\KeywordTok{paste}\NormalTok{(}\StringTok{"Meta de vendas não foi cumprida, com diferença de"}\NormalTok{, diferenca))  }
  
\NormalTok{  \}}
\end{Highlighting}
\end{Shaded}

\begin{verbatim}
## [1] "Meta de vendas cumprida com margem de 14"
\end{verbatim}

\hypertarget{o-loop-for}{%
\subsection{\texorpdfstring{O loop
\texttt{for()}}{O loop for()}}\label{o-loop-for}}

Existem outras formas de loop, mas vamos por enquanto focar no mais
útil, o loop for. Sempre usaremos loops for quando precisarmos realizar
operações elemento por elemento.

Genericamente, um loop for tem a forma:

\begin{Shaded}
\begin{Highlighting}[]
\ControlFlowTok{for}\NormalTok{ (i }\ControlFlowTok{in}\NormalTok{ Lista de índices) \{}
  
\NormalTok{  Expressã}\KeywordTok{o}\NormalTok{(i)}
  
\NormalTok{  \}}
\end{Highlighting}
\end{Shaded}

Se temos uma sequência numérica e queremos saber a soma acumulada até o
i-ésimo elemento, basta montar um loop. Se der tudo certo o último
elemento do vetor \texttt{soma\_acumulada} será 5050.

\begin{Shaded}
\begin{Highlighting}[]
\NormalTok{numeros =}\StringTok{ }\KeywordTok{seq}\NormalTok{(}\DecValTok{1}\NormalTok{, }\DecValTok{100}\NormalTok{)}
\NormalTok{soma_acumulada =}\StringTok{ }\KeywordTok{vector}\NormalTok{() }\CommentTok{# declaramos um vetor vazio}

\ControlFlowTok{for}\NormalTok{(i }\ControlFlowTok{in} \DecValTok{1}\OperatorTok{:}\DecValTok{100}\NormalTok{) \{}
  
\NormalTok{  soma_acumulada[i] =}\StringTok{ }\KeywordTok{sum}\NormalTok{(numeros[}\DecValTok{1}\OperatorTok{:}\NormalTok{i]) }\CommentTok{# preenchemos o vetor vazio}
  \CommentTok{#numeros[1:i] pega todos os elementos de "numeros" entre o primeiro e o i-ésimo}
\NormalTok{\}}

\KeywordTok{print}\NormalTok{(soma_acumulada)}
\end{Highlighting}
\end{Shaded}

\begin{verbatim}
##   [1]    1    3    6   10   15   21   28   36   45   55   66   78   91  105  120
##  [16]  136  153  171  190  210  231  253  276  300  325  351  378  406  435  465
##  [31]  496  528  561  595  630  666  703  741  780  820  861  903  946  990 1035
##  [46] 1081 1128 1176 1225 1275 1326 1378 1431 1485 1540 1596 1653 1711 1770 1830
##  [61] 1891 1953 2016 2080 2145 2211 2278 2346 2415 2485 2556 2628 2701 2775 2850
##  [76] 2926 3003 3081 3160 3240 3321 3403 3486 3570 3655 3741 3828 3916 4005 4095
##  [91] 4186 4278 4371 4465 4560 4656 4753 4851 4950 5050
\end{verbatim}

Observe que vários parâmetros do loop podem se adaptar automaticamente
aos dados com um pouco de imaginação. Vamos acessar dados prontos com a
função \texttt{data()}. Mais especificamente a base iris, com dados de
algumas espécies de flores. A função \texttt{head()} mostra as primeiras
linhas da base para que tenhamos uma ideia do que ela mostra e como está
estruturada.

\begin{Shaded}
\begin{Highlighting}[]
\KeywordTok{data}\NormalTok{(iris)}

\KeywordTok{head}\NormalTok{(iris)}
\end{Highlighting}
\end{Shaded}

\begin{verbatim}
##   Sepal.Length Sepal.Width Petal.Length Petal.Width Species
## 1          5.1         3.5          1.4         0.2  setosa
## 2          4.9         3.0          1.4         0.2  setosa
## 3          4.7         3.2          1.3         0.2  setosa
## 4          4.6         3.1          1.5         0.2  setosa
## 5          5.0         3.6          1.4         0.2  setosa
## 6          5.4         3.9          1.7         0.4  setosa
\end{verbatim}

Digamos que não saibamos quantas variáveis a base tem e queremos fazer
um loop que diga a classe de todos os vetores. Lembre-se que para
acessar elementos específicos de DataFrames e Listas \emph{como eles o
são e não como subconjuntos do DF ou da list} não podemos usar só uma
chave\texttt{{[}{]}}, precisamos usar duas \texttt{{[}{[}{]}{]}}

\begin{Shaded}
\begin{Highlighting}[]
\ControlFlowTok{for}\NormalTok{(i }\ControlFlowTok{in} \DecValTok{1}\OperatorTok{:}\KeywordTok{ncol}\NormalTok{(iris)) \{ }\CommentTok{#ncol() pega o número de colunas de um dataframe}
  
  \KeywordTok{print}\NormalTok{(}\KeywordTok{class}\NormalTok{(iris[[i]]))}
  
\NormalTok{\}}
\end{Highlighting}
\end{Shaded}

\begin{verbatim}
## [1] "numeric"
## [1] "numeric"
## [1] "numeric"
## [1] "numeric"
## [1] "factor"
\end{verbatim}

\hypertarget{exercuxedcios-1}{%
\section{Exercícios}\label{exercuxedcios-1}}

\begin{itemize}
\tightlist
\item
  Gere um DataFrame que é uma ``grade'' 100x100. Você terá então um
  DataFrame com \(100^2\) linhas e duas colunas, cada uma representando
  uma coordenada de um espaço de duas dimensões. Mais explicitamente, a
  primeira coluna terá cem vezes o número 1, cem vezes o número 2 e
  assim em diante, com a segunda tendo cem vezes a sequência de 1 a 100.
\item
  Faça um loop para obter a média de todas as variáveis da base de dados
  \texttt{longley}
\item
  A base \texttt{LifeCycleSavings} contém dados de estrutura etária e
  renda. Localize a documentação da base para descobrir o que é cada
  variável e defina um DataFrame que contém todos os países com taxa de
  crescimento da renda disponível maior do que 3\%.
\item
  A base \texttt{mtcars} contém dados de modelos diferentes de carros e
  todas as suas variáveis são numéricas. Transforme todas as variáveis
  discretas, o número de cilindros, de marchas e de carburadores em
  fatores.
\end{itemize}

\hypertarget{extra-criauxe7uxe3o-de-funuxe7uxf5es}{%
\section{(Extra) Criação de
funções}\label{extra-criauxe7uxe3o-de-funuxe7uxf5es}}

Grande parte da rotina de analisar dados é um tanto quanto repetitiva.
Observe que comandos muito frequentes como loops e gerar sequências têm
funções prontas. Sempre que encararmos uma atividade repetida, é uma boa
ideia criar uma função nova automizando seu procedimento repetido usando
\texttt{function()}. Declaramos um objeto com um nome e a ele passamos a
chamada dessa função primordial. Os argumentos de \texttt{function()}
são justamente os argumentos da função que você está criando, e com
certa flexibilidade. Algumas funções muito simples sequer precisam de
argumentos, como \texttt{getwd()} por exemplo.

\begin{Shaded}
\begin{Highlighting}[]
\NormalTok{mensagem =}\StringTok{ }\ControlFlowTok{function}\NormalTok{() \{}
  \KeywordTok{print}\NormalTok{(}\StringTok{"Insira aqui uma frase motivacional"}\NormalTok{)}
\NormalTok{\}}

\CommentTok{#evocando a função}
\KeywordTok{mensagem}\NormalTok{()}
\end{Highlighting}
\end{Shaded}

\begin{verbatim}
## [1] "Insira aqui uma frase motivacional"
\end{verbatim}

Podemos fazer funções que dependam de argumentos a serem totalmente
especificados. Como normalmente isso envolve algum tipo de manipulação e
solução de algoritimo, queremos devolver ao usuário a solução disso.
Usamos a função \texttt{return()} para quando queremos cortar a execução
e deixamos o R cuidar de devolver ao usuário a última linha da função
quando não.

\begin{Shaded}
\begin{Highlighting}[]
\NormalTok{soma.minha <-}\StringTok{ }\ControlFlowTok{function}\NormalTok{(x, y)\{}
\NormalTok{  x }\OperatorTok{+}\StringTok{ }\NormalTok{y}
\NormalTok{\}}

\KeywordTok{soma.minha}\NormalTok{(}\DecValTok{2}\NormalTok{, }\DecValTok{4}\NormalTok{)}
\end{Highlighting}
\end{Shaded}

\begin{verbatim}
## [1] 6
\end{verbatim}

\begin{Shaded}
\begin{Highlighting}[]
\KeywordTok{soma.minha}\NormalTok{(}\DecValTok{2}\NormalTok{, }\StringTok{"h"}\NormalTok{)}
\end{Highlighting}
\end{Shaded}

\begin{verbatim}
## Error in x + y: non-numeric argument to binary operator
\end{verbatim}

Funções, assim como código normal, aceitam controle de fluxo e isso é
muito útil para lidar com possíveis erros cometidos pelo usuário.
Existem alguns pormenores aqui como por exemplo as funções
\texttt{stop()}, \texttt{message()}, \texttt{warning()} e
\texttt{print()} ou sutilezas como a diferença entre \texttt{library()}
e \texttt{require()} ou mesmo \texttt{require.Namespace()}. Caso vá
escrever funções que serão usadas por mais gente, é importante se
informar sobre essas detalhes mais finos. Uma boa fonte é o livro
Advanced R de Hadley Wickham.

\begin{Shaded}
\begin{Highlighting}[]
\NormalTok{soma.minha =}\StringTok{ }\ControlFlowTok{function}\NormalTok{(x,y) \{}
  
  \ControlFlowTok{if}\NormalTok{(}\OperatorTok{!}\KeywordTok{is.numeric}\NormalTok{(x) }\OperatorTok{|}\StringTok{ }\OperatorTok{!}\KeywordTok{is.numeric}\NormalTok{(y)) \{}

    \KeywordTok{print}\NormalTok{(}\StringTok{"Você deve inserir dois números para que a função possa ser executada"}\NormalTok{)}

\NormalTok{    \} }\ControlFlowTok{else}\NormalTok{ \{}
      
\NormalTok{    x }\OperatorTok{+}\StringTok{ }\NormalTok{y}
  
\NormalTok{    \}}
\NormalTok{\}}

\KeywordTok{soma.minha}\NormalTok{(}\DecValTok{2}\NormalTok{, }\StringTok{"h"}\NormalTok{)}
\end{Highlighting}
\end{Shaded}

\begin{verbatim}
## [1] "Você deve inserir dois números para que a função possa ser executada"
\end{verbatim}

Alguns argumentos não precisam ser especificados \emph{sempre}. Lidando
com dados, estamos quase sempre interessados em alguns limiares de
significância por exemplo, ou quando vamos trabalhar quase sempre com
uma certa taxa de erro aceitável. Para isso especificamos valores padrão
para argumentos.

\begin{Shaded}
\begin{Highlighting}[]
\CommentTok{###argumentos com padrão}

\NormalTok{potencia <-}\StringTok{ }\ControlFlowTok{function}\NormalTok{(x, }\DataTypeTok{base =} \DecValTok{10}\NormalTok{)\{}
\NormalTok{  base}\OperatorTok{^}\NormalTok{x}
\NormalTok{\}}

\KeywordTok{potencia}\NormalTok{(}\DecValTok{3}\NormalTok{)}
\end{Highlighting}
\end{Shaded}

\begin{verbatim}
## [1] 1000
\end{verbatim}

\begin{Shaded}
\begin{Highlighting}[]
\KeywordTok{potencia}\NormalTok{(}\DecValTok{3}\NormalTok{, }\DecValTok{4}\NormalTok{)}
\end{Highlighting}
\end{Shaded}

\begin{verbatim}
## [1] 64
\end{verbatim}

\end{document}
